
% Copyright 2003, 2004 by Till Tantau <tantau@users.sourceforge.net>.
%
% This program can be redistributed and/or modified under the terms
% of the GNU Public License, version 2.


\section{Themes}

\subsection{The Five Different Kinds of Themes}

\emph{Themes} make it easy to change the appearance of a
presentation. The \beamer\ class uses five different kinds of themes:
\begin{description}
\item[Presentation Themes]
  Conceptually, a presentation theme dictates for every single detail
  of a presentation what it looks like. Thus, choosing a particular
  presentation theme will setup for, say, the numbers in enumeration
  what color they have, what color their background has, what font is
  used to render them, whether a circle or ball or rectangle or
  whatever is drawn behind them, and so forth. Thus, when you choose
  a presentation theme, your presentation will look they way someone
  (the creator of the theme) thought that a presentation should look
  like. Presentation themes typically only choose a particular color
  theme, font theme, element theme, and layout theme that go well
  together. 
\item[Color Themes]
  A color theme only dictates which colors are used in a
  presentation. If you have choosen a particular presentation theme
  and then choose a color theme, only the colors of your presentation
  will change. A color theme can specify colors in a very detailed
  way: For example, a color theme can specifically change the colors
  used to render, say, the border of a button, the background of a
  button, and the text on a button.
\item[Font Themes]
  A font theme dictates which fonts or font attributes are used in a
  presentation. As for colors, the font of all text elements used in a
  presentation can be specified independently.
\item[Element Themes]
  An element theme specifies how certain ``elements'' of a
  presentation are rendered. ``Elements'' are everything that needs to
  be typeset and that is not part of the headlines, footlines, and
  sidebars. This includes all enumerations, itemize environments,
  block environments, theorem environments, or the table of
  contents. For each of these, an element theme dictates how these
  ``elements'' should be typeset. For example, an element theme might
  specify that in an enumeration the number should be typeset without
  a dot and that a small circle should be shown behind it. The element
  theme would \emph{not} specify what color should be used for the
  number or the circle (this is the job of the color theme) nor which font
  should be used (this is the job of the font theme).
\item[Layout Themes]
  A layout theme specifies what the layout of the presentation slides
  should look like. Thus, it specifies whether there are head- and
  footlines, what is shown in them, whether there is a sidebar, where
  the logo goes, where the navigation symbols and bars go, and so
  on. It also specifies where the frametitle is put and how it is
  rendered. 
\end{description}

The different themes reside in the four subdirectories |theme|, |color|,
|font|, |elements|, and |layout| of the directory
|beamer/themes|. Internally, a theme is stored as a normal style
file. However, to use a theme, the following special commands should
be used:

\begin{command}{\usetheme\oarg{options}\marg{name}}
  Installs the presentation theme named \meta{name}. Currently, the
  effect of this command is the same as saying |\usepackage| for the
  style file named |beamertheme|\meta{name}|.sty|.
\end{command}


\begin{command}{\usecolortheme\oarg{options}\marg{name}}
  Installs the color theme named \meta{name}. Currently, the effect of this
  command is the same as saying |\usepackage| for the style file named
  |beamercolortheme|\meta{name}|.sty|.
\end{command}

\begin{command}{\usefonttheme\oarg{options}\marg{name}}
  Installs the font theme named \meta{name}. Currently, the effect of this
  command is the same as saying |\usepackage| for the style file named
  |beamerfonttheme|\meta{name}|.sty|.
\end{command}

\begin{command}{\useelementtheme\oarg{options}\marg{name}}
  Installs the element theme named \meta{name}. Currently, the effect of this
  command is the same as saying |\usepackage| for the style file named
  |beamerelementtheme|\meta{name}|.sty|.
\end{command}

\begin{command}{\uselayouttheme\oarg{options}\marg{name}}
  Installs the layout theme named \meta{name}. Currently, the effect of this
  command is the same as saying |\usepackage| for the style file named
  |beamerlayouttheme|\meta{name}|.sty|.
\end{command}

If you do not use any of these commands, a sober \emph{default} theme
is used for all of them. In the following, the presentation, element,
and layout themes that come with the \beamer\ class are described. The
color themes and font themes are discussed in the sections on colors
and fonts, respectively.


\subsection{Presentation Themes}

A presentation theme dictates for every single detail
of a presentation what it looks like. Normally, having chosen a
particular presentation theme, you do not need to specify anything
else having to do with the appearence of your presentation---the
creator of the theme should have taken of the for you. However, you
still \emph{can} change things afterward either by using a different
color, font, element, or even layout theme; or by changing specific
colors, fonts, or templates directly.

When I started naming the presentation themes, I soon ran out of ideas
on how to call them. Instead of giving them more and more cumbersome
names, I decided to switch to a different naming convention:
Except for two exceptions, all presentation themes are named after
cities. These cities happen to be cities in which or near which there
was a conference or workshop that I attended or that a
co-author of mine attended. 


\subsubsection{Presentation Themes Without Navigation Bars}

\begin{themeexample}{default}
As the name suggests, this theme is installed by default. It is a
sober no-nonsense theme that makes minimal use of color or font
variations. This theme is useful for all kinds of talks, except for
very long talks.
\end{themeexample}


\begin{themeexample}[{\opt{|[headheight=|\meta{head height}|,footheight=|\meta{foot height}|]|}}]{boxes}
  For this theme, you can specify an arbitrary number of templates for
  the boxes in the headline and in the footline. You can add a
  template for another box by using the following commands.
\end{themeexample}

\begin{command}{\addheadboxtemplate%
    \marg{background color command}\marg{box template}}
  Each time this command is invoked, a new box is added to the head
  line, with the first added box being shown on the left. All boxes
  will have the same size.
  \example
\begin{verbatim}
\addheadboxtemplate{\color{black}}{\color{white}\tiny\quad 1. Box}
\addheadboxtemplate{\usebeamercolor[bg]{normal text}}
  {\usebeamercolor[fg]{structure}\tiny\quad 2. Box}
\end{verbatim}
\end{command}

\begin{command}{\addfootboxtemplate%
    \marg{background color command}\marg{box template}}
  \example
\begin{verbatim}
\addfootboxtemplate{\color{black}}{\color{white}\tiny\quad 1. Box}
\addheadboxtemplate{\usebeamercolor[bg]{normal text}}
  {\usebeamercolor[fg]{structure}\tiny\quad 2. Box}
\end{verbatim}
\end{command}



\begin{themeexample}{Pittsburgh}
  A sober theme. The right-flushed frame titles cause a certain
  ``tension'' inside each frame.
\end{themeexample}


\begin{themeexample}[\oarg{options}]{Rochester}
  A dominant theme without any navigational elements. Can be made less
  dominant by using a different color theme.

  The following \meta{options} may be given:
  \begin{itemize}
  \item \declare{|height=|\meta{dimension}} sets the height of the
    frame title bar.
  \end{itemize}
\end{themeexample}




\subsubsection{Presentation Themes with a Tree-Like Navigation Bar}

\begin{themeexample}{Antibes}
  A dominant theme with a tree-like navigation at the top. The 
  rectangular elements mirror the rectangular navigation at the
  top. The theme can be made less dominant by using a different color
  theme. 
\end{themeexample}

\begin{themeexample}{JuanLesPins}
  A variation on the |Antibes| theme that has a much ``smoother''
  appearence. It can be made less dominant by chosing a different
  color theme.
\end{themeexample}


\begin{themeexample}{Montpellier}
  A sober theme giving basic navigational hints. The headline can be
  made more dominant by using a different color theme.
\end{themeexample}



\subsubsection{Presentation Themes with a Table of Contents Sidebar}

\begin{themeexample}[\oarg{options}]{Berkeley}
  A dominant theme. If the navigation bar is on the left, it dominates
  since it is seen first. The height of the frame title is fixed to
  two and a half lines, thus you should be careful with overly long
  titles. A logo will be put in the corner area. Rectangular areas
  dominate the layout. The theme can be made less dominant by using a
  different color theme.

  This theme is useful for long talks like lectures that require a
  table of contents to be visible all the time.

  The following \meta{options} may be given:
  \begin{itemize}
  \item \declare{|hideallsubsections|} causes only sections to be
    shown in the sidebar. This is useful, if you need to save
    space.
  \item \declare{|hideothersubsections|} causes only the subsections
    of the current section to be shown. This is useful, if you need to
    save  space.      
  \item \declare{|left|} puts the sidebar on the left (default).
  \item \declare{|right|} puts the sidebar on the right.
  \item \declare{|width=|\meta{dimension}} sets the width of the
    sidebar.
  \end{itemize}
\end{themeexample}

\begin{themeexample}[\oarg{options}]{PaloAlto}
  A variation in the |Berkeley| theme with less dominance of
  rectangular areas. The same \meta{options} as for the |Berkeley|
  theme can be given. 
\end{themeexample}



\begin{themeexample}[\oarg{options}]{Goettingen}
  A relatively sober theme useful for a longer talk that demands a
  sidebar with a full table of contents.

  
  The following \meta{options} may be given:
  \begin{itemize}
  \item \declare{|hideallsubsections|} causes only sections to be
    shown in the sidebar. This is useful, if you need to save
    space.
  \item \declare{|hideothersubsections|} causes only the subsections
    of the current section to be shown. This is useful, if you need to
    save  space.      
  \item \declare{|left|} puts the sidebar on the left (default).
  \item \declare{|right|} puts the sidebar on the right.
  \item \declare{|width=|\meta{dimension}} sets the width of the
    sidebar.
  \end{itemize}
\end{themeexample}

\begin{themeexample}[\meta{options}]{Marburg}
  A very dominat variation of the |Goettingen| theme. The same
  \meta{options} may be given, plus the following option:
  \begin{itemize}
  \item \declare{|tab|} causes the current section or subsection to
    be hilight by moving a tabbing indicator under it, rather
    that hilighting it using a different text color.
  \end{itemize}
\end{themeexample}

\begin{themeexample}[\oarg{options}]{Hannover}
  In this theme, the sidebar on the left is balanced by
  right-flushed frame titles.
    
  The following \meta{options} may be given:
  \begin{itemize}
  \item \declare{|hideallsubsections|} causes only sections to be
    shown in the sidebar. This is useful, if you need to save
    space.
  \item \declare{|hideothersubsections|} causes only the subsections
    of the current section to be shown. This is useful, if you need to
    save  space.      
  \item \declare{|width=|\meta{dimension}} sets the width of the
    sidebar.
  \end{itemize}
\end{themeexample}




\subsubsection{Presentation Themes with a Mini Frame Navigation}

\begin{themeexample}[\oarg{options}]{Berlin}
  A dominant theme with strong colors and dominating rectangular
  areas. The head- and footlines give lot's of information and leave
  little space for the actual slide contents. This theme is useful for
  conferences where the audience is not likely to know the title of
  the talk or who is presenting it.  The theme can be made less
  dominant by using a different color theme.
  
  The following \meta{options} may be given:
  \begin{itemize}
  \item \declare{|compress|} causes the mini frames in the headline to
    use only a single line. This is useful for saving space.
  \end{itemize}
\end{themeexample}

\begin{themeexample}[\oarg{options}]{Ilmenau}
  A variation on the |Berlin| theme. The same \meta{options} may be
  given.  
\end{themeexample}

\begin{themeexample}{Dresden}
  A variation on the |Berlin| theme with a strong separtion into
  navigational stuff at the top/bottom and a sober main text. The same
  \meta{options} may be given.  
\end{themeexample}


\begin{themeexample}{Darmstadt}
  A theme with a strong separation into a navigational upper part and
  an informational main part. By using a different color theme, this
  separation can be lessened. 
\end{themeexample}


\begin{themeexample}{Frankfurt}
  A variaton on the |Darmstadt| theme that is slightly less cluttered
  by leaving out the subsection information.
\end{themeexample}

\begin{themeexample}{Singapore}
  A not-too-sober theme with navigation that does not dominate.
\end{themeexample}

\begin{themeexample}{Szeged}
  A sober theme with a strong dominance of horizontal lines. 
\end{themeexample}




\subsubsection{Presentation Themes with Section and Subsection Tables}

\begin{themeexample}{Copenhagen}
  A not-quite-too-dominant theme. This theme gives compressed
  information about the current section and subsection at the top and
  about the title and the author at the bottom. No shadows are used,
  giving the presentation a ``flat'' look. The theme can be made less
  dominant by using a different color theme.
\end{themeexample}


\begin{themeexample}{Luebeck}
  A variation on the |Copenhagen| theme.
\end{themeexample}

\begin{themeexample}{Malmoe}
  A more sober variation of the |Copenhagen| theme.
\end{themeexample}


\begin{themeexample}{Warsaw}
  A dominant variation of the |Copenhagen| theme.
\end{themeexample}









\subsection{Element Themes}

An element theme installs templates that dictate how the following
``elements'' are typeset:
\begin{itemize}
\item Title and part pages.
\item Itemize environments.
\item Enumerate environments.
\item Descrition environments.
\item Block environments.
\item Theorem and proof environments.
\item Figures and tables.
\item Footnotes.
\item Bibliography entries.
\end{itemize}


\begin{elementthemeexample}{default}
  The default element theme is quite sober. The only extravagance is
  the fact that a little trianlge is used in |itemize| environments
  instead of the usual dot.

  In some cases the theme will honour background color specifications
  for elements. For example, if you set the background color for block
  titles to green, block titles will have a green background. The
  background specifications are honoured for the following elements:
  \begin{itemize}
  \item Title, author, institute, and date fields in the title
    page.
  \item Block environments, both for the title and for the body.
  \end{itemize}
\end{elementthemeexample}

\begin{elementthemeexample}{circles}
  In this theme, |itemize| and |enumerate| items start with a small
  circle. Likewise, entries in the table of contents start with
  circles. 
\end{elementthemeexample}

\begin{elementthemeexample}{rectangles}
  In this theme, |itemize| and |enumerate| items and table of contents
  entries  start with small rectangles. 
\end{elementthemeexample}

\begin{elementthemeexample}[\oarg{options}]{rounded}
  In this theme, |itemize| and |enumerate| items and table of contents
  entries start with small balls. If a background is specified for
  blocks, then the corners of the background rectangles will be
  rounded off. The following \meta{options} may be given:

  \begin{itemize}
  \item \declare{|shadow|} adds a shadow to all blocks.
  \end{itemize}
\end{elementthemeexample}


\subsection{Layout Themes}

\layoutthemeexample{miniframes}
\layoutthemeexample{smoothbars}
\layoutthemeexample{sidebar}
\layoutthemeexample{split}
\layoutthemeexample{shadow}
\layoutthemeexample{tree}
\layoutthemeexample{smoothtree}

%%% Local Variables: 
%%% mode: latex
%%% TeX-master: "beameruserguide"
%%% End: 
