% Copyright 2003, 2004 by Till Tantau <tantau@users.sourceforge.net>.
%
% This program can be redistributed and/or modified under the terms
% of the GNU Public License, version 2.

\section{Structuring a Presentation: The Interactive Global Structure}

\label{section-nonlinear}


\subsection{Adding Hyperlinks and Buttons}

To create anticipated nonlinear jumps in your talk structure, you
can add hyperlinks to your presentation. A hyperlink is a text
(usually rendered as a button) that, when you click on it, jumps the
presentation to some other slide. Creating such a button is a
three-step process: 
\begin{enumerate}
\item
  You specify a target using the command |\hypertarget| or (easier)
  the command |\label|. In some cases, see below, this step may be
  skipped. 
\item
  You render the button using |\beamerbutton| or a similar
  command. This will \emph{render} the button, but clicking it will
  not yet have any effect. 
\item
  You put the button inside a |\hyperlink| command. Now clicking it
  will jump to the target of the link.  
\end{enumerate}

\begin{command}{\hypertarget\sarg{overlay specification}%
    \marg{target name}\marg{text}}
  If the \meta{overlay specification} is present, the \meta{text} is
  the target for hyper jumps to \meta{target name} only on the
  specified slide. On all other slides, the text is shown
  normally. Note that you \emph{must} add an overlay specification to
  the |\hypertarget| command whenever you use it on frames that have
  multiple slides (otherwise |pdflatex| rightfully complains
  that you have defined the same target on different slides).
  \example
\begin{verbatim}
\begin{frame}
  \begin{itemize}
  \item<1-> First item.
  \item<2-> Second item.
  \item<3-> Third item.
  \end{itemize}

  \hyperlink{jumptosecond}{\beamergotobutton{Jump to second slide}}
  \hypertarget<2>{jumptosecond}{}
\end{frame}
\end{verbatim}

  \articlenote
  You must say |\usepackage{hyperref}| in your preamble to use this
  command in |article| mode.
\end{command}

The |\label| command creates a hypertarget as a side-effect and the
|label=|\meta{name} option of the |\frame| command creates a label
named \meta{name}|<|\meta{slide number}|>| for each slide of the frame
as a side-effect. Thus the above example could be written more easily
as: 
\begin{verbatim}
\begin{frame}[label=threeitems]
  \begin{itemize}
  \item<1-> First item.
  \item<2-> Second item.
  \item<3-> Third item.
  \end{itemize}

  \hyperlink{threeitems<2>}{\beamergotobutton{Jump to second slide}}
\end{frame}
\end{verbatim}



The following commands can be used to specify in an abstract way what
a button will be used for.

\begin{command}{\beamerbutton\marg{button text}}
  Draws a button with the given \meta{button text}.
  \example |\hyperlink{somewhere}{\beamerbutton{Go somewhere}}|

  \articlenote
  This command (and the following) just insert their argument in
  |article| mode.

  \begin{element}{button}\yes\yes\yes
    When the |\beamerbutton| command is called, this template is used
    to render the button. Inside the template you can use the command
    |\insertbuttontext| to insert the argument that was passed to
    |\beamerbutton|.
    \begin{templateoptions}
      \itemoption{default}{}
      Typesets the button with rounded corners. The fore- and
      background of the \beamer-color |button| are used and also the
      \beamer-font |button|. The border of the button gets the
      foreground of the \beamer-color |button border|.
    \end{templateoptions}
    The following inserts are useful for this element:
    \begin{itemize}
      \iteminsert{\insertbuttontext} inserts the text of the current
      button. Inside ``Goto-Buttons'' (see below) this text is
      prefixed by the insert |\insertgotosymbol| and similarly for
      skip and return buttons.

      \iteminsert{\insertgotosymbol} This text is inserted at the
      beginning of goto buttons. Redefine this command to change the
      symbol.
      \example
      |\renewcommand{\insertgotosymbol}{\somearrowcommand}|

      \iteminsert{\insertskipsymbol} This text is inserted at the
      beginning of skip buttons.

      \iteminsert{\insertreturnsymbol} This text is inserted at the
      beginning of return buttons.
    \end{itemize}
  \end{element}

  \begin{element}{button border}\no\yes\no
    The foreground of this color is used to render the border of
    buttons. 
  \end{element}
\end{command}

\begin{command}{\beamergotobutton\marg{button text}}
  Draws a button with the given \meta{button text}. Before the text, a
  small symbol (usually a right-pointing arrow) is inserted that
  indicates that pressing this button will jump to another ``area'' of
  the presentation.

  \example |\hyperlink{detour}{\beamergotobutton{Go to detour}}|
\end{command}

\begin{command}{\beamerskipbutton\marg{button text}}
  The symbol drawn for this button is usually a double right
  arrow. Use this button if pressing it will skip over a
  well-defined part of your talk.

  \example
\begin{verbatim}
\frame{
  \begin{theorem}
    ...
  \end{theorem}

  \begin{overprint}
  \onslide<1>
    \hfill\hyperlinkframestartnext{\beamerskipbutton{Skip proof}}
  \onslide<2>
    \begin{proof}
      ...
    \end{proof}
  \end{overprint}
}
\end{verbatim}
\end{command}

\begin{command}{\beamerreturnbutton\marg{button text}}
  The symbol drawn for this button is usually a left-pointing
  arrow. Use this button if pressing it will return from a detour. 

  \example
\begin{verbatim}
\frame<1>[label=mytheorem]
{
  \begin{theorem}
    ...
  \end{theorem}

  \begin{overprint}
  \onslide<1>
    \hfill\hyperlink{mytheorem<2>}{\beamergotobutton{Go to proof details}}
  \onslide<2>
    \begin{proof}
      ...
    \end{proof}
    \hfill\hyperlink{mytheorem<1>}{\beamerreturnbutton{Return}}
  \end{overprint}
}
\appendix
\againframe<2>{mytheorem}
\end{verbatim}
\end{command}

To make a button ``clickable'' you must place it in a command like
|\hyperlink|. The command |\hyperlink| is a standard command of the
|hyperref| package. The \beamer\ class defines a whole bunch of other
hyperlink commands that you can also use.

\begin{command}{\hyperlink\sarg{overlay specification}\marg{target
      name}\marg{link text}\sarg{overlay specification}}
  Only one \meta{overlay specification} may be given.
  The \meta{link text} is typeset in the usual way. If you click
  anywhere on this text, you will jump to the slide on which the
  |\hypertarget| command was used with the parameter \meta{target
    name}. If an \meta{overlay specification} is present, the
    hyperlink (including the \meta{link text}) is completely
    suppressed on the non-specified slides.
\end{command}

The following commands have a predefined target; otherwise they behave
exactly like |\hyperlink|. In particular, they all also accept an
overlay specification and they also accept it at the end, rather than
at the beginning.

\begin{command}{\hyperlinkslideprev\sarg{overlay specification}\marg{link text}}
  Clicking the text jumps one slide back.
\end{command}

\begin{command}{\hyperlinkslidenext\sarg{overlay specification}\marg{link text}}
  Clicking the text jumps one slide forward.
\end{command}
  
\begin{command}{\hyperlinkframestart\sarg{overlay specification}\marg{link text}}
  Clicking the text jumps to the first slide of the current frame.
\end{command}

\begin{command}{\hyperlinkframeend\sarg{overlay specification}\marg{link text}}
  Clicking the text jumps to the last slide of the current frame.
\end{command}

\begin{command}{\hyperlinkframestartnext\sarg{overlay specification}\marg{link text}}
  Clicking the text jumps to the first slide of the next frame.
\end{command}

\begin{command}{\hyperlinkframeendprev\sarg{overlay specification}\marg{link text}}
  Clicking the text jumps to the last slide of the previous frame.
\end{command}

The previous four command exist also with ``|frame|'' replaced by
``|subsection|'' everywhere, and also again with  ``|frame|'' replaced
by ``|section|''.

\begin{command}{\hyperlinkpresentationstart\sarg{overlay specification}\marg{link text}}
  Clicking the text jumps to the first slide of the presentation.
\end{command}

\begin{command}{\hyperlinkpresentationend\sarg{overlay specification}\marg{link text}}
  Clicking the text jumps to the last slide of the presentation. This
  \emph{excludes} the appendix.
\end{command}

\begin{command}{\hyperlinkappendixstart\sarg{overlay specification}\marg{link text}}
  Clicking the text jumps to the first slide of the appendix. If there
  is no appendix, this will jump to the last slide of the document.
\end{command}

\begin{command}{\hyperlinkappendixend\sarg{overlay specification}\marg{link text}}
  Clicking the text jumps to the last slide of the appendix.
\end{command}

\begin{command}{\hyperlinkdocumentstart\sarg{overlay specification}\marg{link text}}
  Clicking the text jumps to the first slide of the presentation.
\end{command}

\begin{command}{\hyperlinkdocumentend\sarg{overlay specification}\marg{link text}}
  Clicking the text jumps to the last slide of the presentation or, if
  an appendix is present, to the last slide of the appendix.
\end{command}




\subsection{Repeating a Frame at a Later Point}

Sometimes you may wish some slides of a frame to be shown in your main
talk, but wish some ``supplementary'' slides of the frame to be shown
only in the the appendix. In this case, the |\againframe| commands is
useful. 


\begin{command}{\againframe\sarg{overlay
      specification}\opt{|[<|\meta{default overlay specification}|>]|}\oarg{options}\marg{name}}
  \beamernote
  Resumes a frame that was previously created using |\frame|
  with the option |label=|\meta{name}. You must have used this option,
  just placing a label inside a frame ``by hand'' is not enough. You
  can use this command to ``continue'' a frame that has been
  interrupted by another frame. The effect of this command is to call
  the |\frame| command with the given \meta{overlay specification},
  \meta{default overlay specification} (if present), and
  \meta{options} (if present) and with the original frame's contents. 

  \example
\begin{verbatim}
\frame<1-2>[label=myframe]
{
  \begin{itemize}
  \item<alert@1> First subject.
  \item<alert@2> Second subject.
  \item<alert@3> Third subject.
  \end{itemize}
}

\frame
{
  Some stuff explaining more on the second matter.
}

\againframe<3>{myframe}
\end{verbatim}
  The effect of the above code is to create four slides. In the first
  two, the items 1 and~2 are hilighted. The third slide contains the
  text ``Some stuff explaining more on the second matter.'' The fourth
  slide is identical to the first two slides, except that the third
  point is now hilighted.

  \example
\begin{verbatim}
\frame<1>[label=Cantor]
{
  \frametitle{Main Theorem}

  \begin{Theorem}
    $\alpha < 2^\alpha$ for all ordinals~$\alpha$.
  \end{Theorem}

  \begin{overprint}
  \onslide<1>
    \hyperlink{Cantor<2>}{\beamergotobutton{Proof details}}

  \onslide<2->
    % this is only shown in the appendix, where this frame is resumed.
    \begin{proof}
      As shown by Cantor, ...
    \end{proof}

    \hfill\hyperlink{Cantor<1>}{\beamerreturnbutton{Return}}
  \end{overprint}
}

...
\appendix

\againframe<2>{Cantor}
\end{verbatim}
  In this example, the proof details are deferred to a slide in the
  appendix. Hyperlinks are setup, so that one can jump to the proof
  and go back.

  \articlenote
  This command is ignored in |article| mode.

  \lyxnote
  Use the style ``AgainFrame'' to insert an |\againframe| command. The
  \meta{label name} is the text on following the style name
  and is \emph{not} put in \TeX-mode. However, an overlay specification
  must be given in \TeX-mode and it must precede the label name.
\end{command}



\subsection{Adding Anticipated Zooming}

\label{section-zooming}


Anticipated zooming is necessary when you have a very complicated
graphic that you are not willing to simplify since, indeed, all the
complex details merit an explanation. In this case, use the command
|\framezoom|. It allows you to specify that clicking on a certain area
of a frame should zoom out this area. You can then explain the
details. Clicking on the zoomed out picture will take you back to the
original one. 

\begin{command}{\framezoom\ssarg{button overlay
      specification}\ssarg{zoomed overlay
      specification}\oarg{options}\\|(|\meta{upper left x}|,|\meta{upper
      left y}|)(|\meta{zoom area width}|,|\meta{zoom area depth}|)|}
  This command should be given somewhere at the beginning of a
  frame. When given, two different things will happen, depending on
  whether the \meta{button overlay specification} applies to the
  current slide of the frame or whether the \meta{zoomed overlay
    specification} applies. These overlay specifications should not
  overlap.

  If the \meta{button overlay specification} applies, a clickable are
  is created inside the frame. The size of this area is given by
  \meta{zoom area width} and \meta{zoom area depth}, which are two
  normal \TeX\ dimensions (like |1cm| or |20pt|). The upper left
  corner of this area is given by \meta{upper left x} and \meta{upper
  left y}, which are also \TeX\ dimensions. They are measures
  \emph{relative to the place where the first normal text of a
    frame would go}. Thus, the location |(0pt,0pt)| is at the
  beginning of the normal text (which excludes the headline and also
  the frame title).

  By default, the button is clickable, but it will not be indicated in
  any special way. You can draw a border around the button by using
  the following \meta{option}:
  \begin{itemize}
  \item \declare{|border|}\opt{|=|\meta{width in pixels}} will draw
    a border around the specified button area. The default width is 1
    pixel. The color of this  button is the |linkbordercolor| of
    |hyperref|. \beamer\ sets this color to a 50\% gray by default. To
    change this, you can use the command
    |\hypersetup{linkbordercolor={|\meta{red}| |\meta{green}| |\meta{blue}|}}|, 
    where \meta{red}, \meta{green}, and \meta{blue} are values between
    0 and 1.
  \end{itemize}

  When you press the button created in this way, the viewer
  application will hyperjump to the first of the frames specified by
  the \meta{zoomed overlay specification}. For the slides to which
  this overlay specification applies, the following happens:

  The exact same area as the one specified before is ``zoomed out'' to
  fill the whole normal text area of the frame. Everything else,
  including the sidebars, the headlines and footlines, and even the
  frame title retain their normal size. The zooming is performed in
  such a way that the whole specified area is completely shown. The
  aspect ratio is kept correct and the zoomed area will possibly show
  more than just the specified area if the aspect ratio of this area
  and the aspect ratio of the available text area do not agree.

  Behind the whole text area (which contains the zoomed area) a big
  invisible ``Back'' button is put. Thus clicking anywhere on the text
  area will jump back to the original (unzoomed) picture.

  You can specify several zoom areas for a single frame. In this case,
  you should specify different \meta{zoomed overlay specification},
  but you can specify the same \meta{button overlay
  specification}. You cannot nest zoomings in the sense that you
  cannot have a zoom button on a slide that is in some \meta{zoomed
  overlay specification}. However, you can have overlapping and even
  nested \meta{button overlay specification}. When clicking on an area
  that belongs to several buttons, the one given last will ``win'' (it
  should hence be the smallest one).

  If you do not wish to have the frame title shown on a zoomed slide,
  you can add an overlay specification to the |\frametitle| command
  that simply suppresses the title for the slide. Also, by using the
  |plain| option, you can have the zoomed slide fill the whole page. 

  \example A simple case
\begin{verbatim}
\begin{frame}
  \frametitle{A Complicated Picture}

  \framezoom<1><2>(0cm,0cm)(2cm,1.5cm)
  \framezoom<1><3>(1cm,3cm)(2cm,1.5cm)
  \framezoom<1><4>(3cm,2cm)(3cm,2cm)

  \pgfimage[height=8cm]{complicatedimagefilename}
\end{frame}
\end{verbatim}

  \example A more complicate case in which the zoomed parts completely
  fill the frames.
\begin{verbatim}
\begin{frame}<1>[label=zooms]
  \frametitle<1>{A Complicated Picture}

  \framezoom<1><2>[border](0cm,0cm)(2cm,1.5cm)
  \framezoom<1><3>[border](1cm,3cm)(2cm,1.5cm)
  \framezoom<1><4>[border](3cm,2cm)(3cm,2cm)

  \pgfimage[height=8cm]{complicatedimagefilename}
\end{frame}
\againframe<2->[plain]{zooms}
\end{verbatim}
\end{command}



\subsection{Using the Navigation Bars}
\label{section-navigation-bars}

Navigation bars and symbols are two independent concepts that can be
used to navigate through a presentation. They are created
automatically. Most themes that come along with the \beamer\ class
show some kind of navigation bar during your talk. Although these
navigation bars take up quite a bit of space, they are often useful
for two reasons: 

\begin{itemize}
\item
  They provide the audience with a visual feedback of how much of your
  talk you have covered and what is yet to come. Without such
  feedback, an audience will often puzzle whether something you are
  currently introducing will be explained in more detail later on or
  not.
\item
  You can click on all parts of the navigation bar. This will directly
  ``jump'' you to the part you have clicked on. This is particularly
  useful to skip certain parts of your talk and during a ``question
  session,'' when you wish to jump back to a particular frame someone
  has asked about.
\end{itemize}

Some navigation bars can be ``compressed'' using the following option:

\begin{classoption}{compress}
  Tries to make all navigation bars as small as possible. For example,
  all small frame representations in the navigation bars for a single
  section are shown alongside each other. Normally, the representations
  for different subsections are shown in different lines. Furthermore,
  section and subsection navigations are compressed into one line.
\end{classoption}

When you click on one of the icons representing a frame in a
navigation bar (by default this is icon is a small circle), the
following happens: 
\begin{itemize}
\item
  If you click on (the icon of) any frame other than the current frame, the
  presentation will jump to the first slide of the frame you clicked
  on.
\item
  If you click on the current frame and you are not on the last slide
  of this frame, you will jump to the last slide of the frame.
\item
  If you click on the current frame and you are on the last slide, you
  will jump to the first slide of the frame.
\end{itemize}
By the above rules you can:
\begin{itemize}
\item
  Jump to the beginning of a frame from somewhere else by clicking on
  it once.  
\item
  Jump to the end of a frame from somewhere else by clicking on it
  twice.
\item
  Skip the rest of the current frame by clicking on it once.
\end{itemize}

I also tried making a jump to an already-visited frame jump
automatically to the last slide of this frame. However, this turned
out to be more confusing than helpful. With the current implementation
a double-click always brings you to the end of a slide, regardless
from where you ``come.''

By clicking on a section or subsection in the navigation bar, you will
jump to that section. Clicking on a section is particularly useful if
the section starts with a |\tableofcontents[currentsection]|, since you
can use it to jump to the different subsections.

By clicking on the document title in a navigation bar (not all themes
show it), you will jump to the first slide of your presentation
(usually the title page) \emph{except} if you are already at the first
slide. On the first slide, clicking on the document title will jump to
the end of the presentation, if there is one. Thus by \emph{double}
clicking the document title in a navigation bar, you can jump to the end.



\subsection{Using the Navigation Symbols}
\label{section-navigation-symbols}

Navigation symbols are small icons that are shown on every slide
by default. The following symbols are shown: 
\begin{enumerate}
\item
  A slide icon, which is depicted as  a single rectangle. To the left and
  right of this symbol, a left and right arrow are shown.
\item
  A frame icon, which is depicted as three slide icons ``stacked on top of
  each other''. This symbol is framed by arrows.
\item
  A subsection icon, which is depicted as a highlighted subsection
  entry in a table of contents. This  symbols is framed by arrows.
\item
  A section icon, which is depicted as a highlighted section entry
  (together with all subsections) in a table of contents. This symbol
  is framed by arrows.
\item
  A presentation icon, which is depicted as a completely highlighted
  table of contents.
\item
  An appendix icon, which is depicted as a completely highlighted
  table of contents consisting of only one section. (This icon is only
  shown if there is an appendix.)
\item
  Back and forward icons, depicted as circular arrows.
\item
  A ``search'' or ``find'' icon, depicted as a detective's
  magnifying glass.
\end{enumerate}

Clicking on the left arrow next to an icon always jumps to (the
last slide of) the previous slide, frame, subsection, or
section. Clicking on the right arrow next to an icon always jump to
(the first slide of) the next slide, frame, subsection, or section. 

Clicking \emph{on} any of these icons has different effects:
\begin{enumerate}
\item
  If supported by the viewer application, clicking on a slide icon
  pops up a window that allows you to enter a slide number to which
  you wish to jump.
\item
  Clicking on the left side of a frame icon will jump to the first
  slide of the frame, clicking on the right side will jump to the last
  slide of the frame (this can be useful for skipping overlays).
\item
  Clicking on the left side of a subsection icon will jump to the
  first slide of the subsection, clicking on the right side will jump
  to the last slide of the subsection.
\item
  Clicking on the left side of a section icon will jump to the
  first slide of the section, clicking on the right side will jump
  to the last slide of the section.
\item
  Clicking on the left side of the presentation icon will jump to the
  first slide, clicking on the right side will jump to the last slide
  of the presentation. However, this does \emph{not} include the
  appendix. 
\item
  Clicking on the left side of the appendix icon will jump to the
  first slide of the appendix, clicking on the right side will jump to
  the last slide of the appendix.
\item
  If supported by the viewer application, clicking on the back and
  forward symbols jumps to the previously visited slides.
\item
  If supported by the viewer application, clicking on the search icon
  pops up a window that allows you to enter a search string. If found,
  the viewer application will jump to this string.
\end{enumerate}

You can reduce the number of icons that are shown or their layout by
adjusting the |navigation symbols| template.


\begin{element}{navigation symbols}\yes\yes\yes
  This template is invoked in ``three-star-mode'' by themes
  at the place where the navigation symbols should be
  shown. ``Three-star-mode'' means that the command
  |\usebeamertemplate***| is used.

  Note that, although it may \emph{look} like that the symbols are part of
  the footline, they are more often part of an invisible right
  sidebar.

  \begin{templateoptions}
    \itemoption{default}{}
    Organizes the navigation symbols horizontally.
    \itemoption{horizontal}{}
    This is an alias for the default.
    \itemoption{vertical}{}
    Organizes the navigation symbols vertically.
    \itemoption{only frame symbol}{}
    Shows only the navigational symbol for navigating frames.
  \end{templateoptions}

  \example The following command suppresses all navigation symbols:
\begin{verbatim}
\setbeamertemplate{navigation symbols}{}
\end{verbatim}

  Inside this template, the following inserts are useful:
  \begin{itemize}
    \iteminsert{\insertslidenavigationsymbol}
    Inserts the slide navigation symbols, that is, the slide symbols
    (a rectangle) together with arrows to the left and right that are
    hyperlinked.

    \iteminsert{\insertframenavigationsymbol}
    Inserts the frame navigation symbol.

    \iteminsert{\insertsubsectionnavigationsymbol}
    Inserts the subsection navigation symbol.

    \iteminsert{\insertsectionnavigationsymbol}
    Inserts the section navigation symbol.

    \iteminsert{\insertdocnavigationsymbol}
    Inserts the presentation navigation symbol and (if necessary) the
    appendix navigation symbol.

    \iteminsert{\insertbackfindforwardnavigationsymbol}
    Inserts a back, a find, and a forward navigation symbol.
  \end{itemize}
\end{element}








%%% Local Variables: 
%%% mode: latex
%%% TeX-master: "beameruserguide"
%%% End: 
