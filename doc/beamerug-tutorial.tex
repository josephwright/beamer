% Copyright 2003, 2004 by Till Tantau <tantau@users.sourceforge.net>.
%
% This program can be redistributed and/or modified under the terms
% of the GNU Public License, version 2.

\section{A Short Tutorial: Euclid's Presentation}
\label{section-tutorial}

This section presents a short tutorial that focuses on those features
of \beamer\ that you are likely to be interested in at the
beginning and leaves out all the glorious details that are explained
in great detail later on.

We wish to help Prof.\ Euclid of the University of Alexandria to
create a presentation on his latest discovery: There are 
infinitely many prime numbers! Euclid wrote a paper on this and it got
accepted at the 27th International Symposium on Prime Numbers $-280$
(ISPN~'80). Euclid wishes to use the \beamer\ class
to create a presentation for the conference.
On the conference webpage he found out that he will have twenty
minutes for his talk, including questions.

The first things Euclid should do is to look for a solution template
for his presentation. Having a look at
Section~\ref{section-solutions}, he finds that the file
\begin{verbatim}
beamer/solutions/conference-talks/conference-ornate-20min.en.tex
\end{verbatim}
might be appropriate. He creates a subdirectories |presentation| in
the directory that contains the actual paper and copies the solution
template to this subdirectory, renaming to |main.tex|.

\lyxnote
If Euclid uses \LyX, he would choose ``New from template'' and pick
the template file 
\begin{verbatim}
beamer/solutions/conference-talks/conference-ornate-20min.en.lyx
\end{verbatim}

Then he opens the file in his favorite editor. The file starts with
\begin{verbatim}
\documentclass{beamer}
\end{verbatim}
which Euclid finds hardly surprising. Next comes a line reading 
\begin{verbatim}
\mode<presentation>
\end{verbatim}
which Euclid does not understand. Since he finds more stuff in the file
that he does not understand, he decides to ignore all of that for time
being, hoping that it all serves some good purpose.

The next thing that seems logical is the place where the |\title|
command is used. Naturally, he replaces it with
\begin{verbatim}
\title{There Is No Largest Prime Number}
\end{verbatim}
since this was the title of the paper. He sees that the command
|\title| also takes an optional ``short'' argument in square brackets,
which is shown in places where there is little space, but he decides
that the title is short enough by itself.

Euclid next adjusts the |\author| and |\date| fields as follows:
\begin{verbatim}
\author{Euclid of Alexandria}
\date[ISPN '80]{27th International Symposium of Prime Numbers}
\end{verbatim}
For the date, he felt that the name was a little long, so a short
version should be given (|ISPN '80|).

There are two fields that Euclid does not know, but whose meaning he
can guess: |\subtitle| and |\institute|. He adjusts them
accordingly. (Euclid does not need to use the |\and| command, which is
used to separate several authors, nor the |\inst| command, which just
makes its argument a superscript).

\lyxnote
In \LyX, Euclid just edits the first lines having of the different
styles like Author or Title or Date. He deletes the optional short
fields.


The next thing in the file that seems interesting is where the first
``frame'' is created, right after the |\||begin{document}|:

\begin{verbatim}
\begin{frame}
  \titlepage
\end{frame}
\end{verbatim}

In \beamer, a presentation consists of a series of frames. Each frame
in turn may consist of several slides (if there is more than one, they
are called overlays). Normally, everything between |\begin{frame}|
  and |\end{frame}| is put on a single slide. No page breaking is
performed.

\lyxnote
The title page frame is created automatically by \LyX. All other
frames start with the style BeginFrame and end either with the style
EndFrame or, automatically, with the start of the next frame,
subsection, or section. 

Euclid sees that the first frame is ``filled'' by the title page,
which seems quite logical. Eager to find out how the first page will
look, he invokes |pdflatex| on his file |main.tex| (twice). Then he
uses the Acrobat Reader or |xpdf| to view the resulting
|main.pdf|. Indeed, the first page contains all the information Euclid
has provided until now. It even looks quite impressive with the
colorful title and the rounded corners and the shadows, but he is
doubtful whether he should leave it like that. He decides to address
this problem later.

\lyxnote
Euclid chooses View $\to$ PDF to view the resulting presentation. On a
slow machine, this may take a while. See Section~\ref{section-speedup}
for ways of speeding up the compilation.

The next frame contains a table of contents:
\begin{verbatim}
\begin{frame}
  \frametitle{Outline}
  \tableofcontents
\end{frame}
\end{verbatim}
Furthermore, this frame has an individual title (Outline). A comment
in the frame says that Euclid might wish to try to add the
|[pausesections]| option. He tries this, changing the frame to:
\begin{verbatim}
\begin{frame}
  \frametitle{Outline}
  \tableofcontents[pausesections]
\end{frame}
\end{verbatim}
After re-pdf\LaTeX ing the presentation, he finds that instead of a
single slide, there are now two ``table of contents slides'' in the
presentation. On the first of these, only the first section is shown,
on the second both sections are shown (scanning down in the file,
Euclid finds that, indeed, there are |\section| commands introducing
these sections). The effect of the |pausesections| seems to be that
one can talk about the first section before the second one is
shown. Then, Euclid can press the down- or right-key, to show the
complete table of contents and can talk about the second section.

The next commands Euclid finds are
\begin{verbatim}
\section{Motivation}
\subsection{The Basic Problem That We Studied}
\end{verbatim}
These commands are given \emph{outside} of frames. So Euclid assumes
that at the point of invocation they have no direct effect, they only
create entries in the table of contents. Having a ``Motivation''
section seems reasonable to Euclid, but he changes the |\subsection|
title.

As he looks at the presentation, he notices that his assumption was
not quite true: each |\subsection| command seems to insert a frame
containing a table of contents into the presentation. Doubling back he
finds the command that causes this: The |\AtBeginSubsection| inserts a
frame with only the current subsection hilighted at the beginning of
each section. If Euclid does not like this, he can just delete the
whole |\AtBeginSubsection| stuff and the table of contents at the
beginning of each subsection disappear. 

The |\section| and |\subsection| commands take optional short
arguments. These short arguments are used whenever a short form of the
section of subsection name is needed. While this is in keeping with
the way \beamer\ treats the optional arguments of things like
|\title|, it is \emph{different} from the usual way \LaTeX\ treats an
optional argument for sections (where the optional argument dictates
what is shown in the table of contents and the main argument dictates
what is shown everywhere else; in \beamer\ things are exactly the
other way round).

Euclid then modifies the next frame, which is the first ``real'' frame
of the presentation, as follows:
\begin{verbatim}
\begin{frame}
  \frametitle{What Are Prime Numbers?}
  A prime number is a number that has exactly two divisors. 
\end{frame}
\end{verbatim}
This yields the desired result. It might be a good idea to put some
emphasis on the object being defined (prime numbers). Euclid tries
|\emph| but finds that too mild an emphasis. \beamer\ offers the
command |\alert|, which is used like |\emph| and, by default, typesets
its argument in bright red.
\lyxnote
The |\alert| command needs to be entered in \TeX-mode, which is
awkward. It's easier to just paint the text in red.

Next, Euclid decides to make it even clearer that he is giving a
definition by putting a |definition| environment around the
definition. 
\begin{verbatim}
\begin{frame}
  \frametitle{What Are Prime Numbers?}
  \begin{definition}
    A \alert{prime number} is a number that has exactly two divisors.
  \end{definition}
\end{frame}
\end{verbatim}
Other useful environments like |theorem|, |lemma|, |proof|,
|corollary|, or |example| are also predefined by \beamer. As in
|amsmath|, they take optional arguments that they show in
brackets. Indeed, |amsmath| is automatically loaded by \beamer.

Since it is always a good idea to add examples, Euclid decides to add
one:
\begin{verbatim}
\begin{frame}
  \frametitle{What Are Prime Numbers?}
  \begin{definition}
    A \alert{prime number} is a number that has exactly two divisors.
  \end{definition}
  \begin{example}
    \begin{itemize}
    \item 2 is prime (two divisors: 1 and 2).
    \item 3 is prime (two divisors: 1 and 3).
    \item 4 is not prime (two divisors: 1, 2, and 4).
  \end{example}
\end{frame}
\end{verbatim}

The frame already looks quite nice, though, perhaps a bit colorful.
However, Euclid would now like to show the three items one after
another, not all three right away. By showing them incrementally, he
hopes to focus the audience's attention on the item he is currently
talking about. He can achieve this by adding \emph{overlay
specifications} to the |\item| commands:
\begin{verbatim}
  \begin{itemize}
    \item<1-> 2 is prime (two divisors: 1 and 2).
    \item<2-> 3 is prime (two divisors: 1 and 3).
    \item<3-> 4 is not prime (two divisors: 1, 2, and 4).
  \end{example}
\end{verbatim}
These overlay specifications are given in pointed brackets. The
specification |<1->| means ``from slide 1 on.'' Thus, the first item
is shown on the first slide of the frame, but the other two items are
not shown. Rather, the second point is shown only from the second
slide onward. \beamer\ automatically computes the number of slides
needed for each frame. More generally, overlay specification are lists
of numbers or number ranges where the start or ending of a range can
be left open. For example |-3,5-6,8-| means ``on all slides, except
for slides 4 and~7.''

\lyxnote
You add overlay specifications to the items by entering \TeX-mode
(press on the little \TeX\ icon) and writing |<1->|. This \TeX-text
should be placed right at the beginning of the item.



Moving on, Euclid creates a frame showing his main argument:
\begin{verbatim}
\begin{frame}
  \frametitle{There Is No Largest Prime Number}
  \framesubtitle{The proof uses \textit{reductio ad absurdum}.}

  \begin{theorem}
    There is no largest prime number.
  \end{theorem}
  \begin{proof}
    \begin{enumerate}
    \item<1-> Suppose $p$ were the largest prime number.
    \item<2-> Let $q$ be the product of the first $p$ numbers.
    \item<3-> Then $q + 1$ is not divisible by any of them.
    \item<1-> Thus $q + 1$ is also prime and greater than $p$.\qedhere
    \end{enumerate}      
  \end{proof}
\end{frame}
\end{verbatim}
Euclid has added the overlay specification |<1->| (and not |<4->|) on
purpose. This causes the last and first points to be shown at the same
time, namely from the first slide onward. He can thus talk about his
new proof idea, namely proof by contradiction, where a wrong
assumption is brought to a contradiction at the end after a number of
intermediate steps that are not important at the beginning.

The |\qedhere| is used to put the \textsc{qed} symbol at the end of
the line \emph{inside} the enumeration. Normally, the \textsc{qed}
symbol is automatically inserted at the end of a proof environment,
but that would be on an ugly empty line here.

The |\item| command is not the only command that takes overlay
specifications. Another useful command that takes one is the
|\uncover| command, which Euclid uses in a line that he adds after the
proof:
\begin{verbatim}
  \uncover<4->{The proof used \textit{reductio ad absurdum}.}
\end{verbatim}
The |\uncover| command only shows its argument on the slides specified
in the overlay specification. On all other slides, the argument is
hidden (though it still occupies space). The command |\only| is
similar and Euclid could also have tried
\begin{verbatim}
  \only<4->{The proof used \textit{reductio ad absurdum}.}
\end{verbatim}
On non-specified slides the |\only| command simply ``throws its
argument away'' and the argument does not occupy any space. This leads
to different heights of the text on the first three slides and of the
fourth slide. If the text is centered vertically, this will cause the
text to ``wobble'' and thus |\uncover| should be used. However, you
sometimes wish things to ``really disappear'' on some slides and then
|\only| is useful. Euclid could also have used the class option
|slidestop|, which causes the text in frames to be vertically flushed
to the top. Then a differing text height does not cause
wobbling. Vertical flushing can also be achieved for only a single
frame be by giving the optional argument |[t]| like this to the
|frame| environment as in 
\begin{verbatim}
\begin{frame}[t]
  \frametitle{There Is No Largest Prime Number}
  ...
\end{frame}
\end{verbatim}
Vice versa, if the |slidestop| option is given, a frame can be
vertically centered using the |[c]| option for the frame.




%%% Local Variables: 
%%% mode: latex
%%% TeX-master: "beameruserguide"
%%% End: 
