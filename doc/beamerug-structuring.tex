% Copyright 2003, 2004 by Till Tantau <tantau@users.sourceforge.net>.
%
% This program can be redistributed and/or modified under the terms
% of the GNU Public License, version 2.

\section{Structuring a Presentation}



\subsection{Creating the Static Global Structure}

This section lists the commands that are used for structuring a
presentation ``globally'' using commands like |\section| or
|\part|. These commands are use to create a \emph{static} structure,
meaning that the resulting presentation is normally presented one slide
after the other in the order the slides occur. 
Section~\ref{section-nonlinear} explains which commands can be used to
create the \emph{interactive} structure. For the interactive structure,
you must interact with the presentation program, typically by clicking
on hyperlinks, to advance the presentation. 

\subsubsection{Adding a Title Page}

You can use the |\titlepage| command to insert a title page into
a frame. By default, it will arrange the following elements on
the title page: the document title, the author(s)'s names, their
affiliation, a title graphic, and a date.

\begin{command}{\titlepage}
  Inserts the text of a title page into the current frame.
  \example |\frame{\titlepage}|
  \example |\frame[plain]{\titlepage}| for a titlepage that fills the
  whole frame.

  \lyxnote
  If you use the ``Title'' style in your presentation, a title page is
  automatically inserted.

  \begin{element}{title page}\yes\yes\yes
    This template is invoked when the |\titlepage| command is used.

    \begin{templateoptions}
      \itemoption{default}{\oarg{orientation}}
      The title page is typeset showing the title, followed by the
      author, his or her affiliation, the date, and a titlegraphic. If
      any of these are missing, they are not shown. Except for the
      titlegraphic, if the \beamer-color |title|, |author|, |institute|,
      or |date| is defined, respectively, it is used to as textcolor for
      these entries. If a background color is defined for them, a
      colored bar in the corresponding color is drawn behind them,
      spanning the text width. The corresponding \beamer-fonts are used
      for these entries.

      The \meta{orientation} option is passed on the |beamercolorbox|
      and can be used, for example, to flush the title page to the left
      by specifying |left| here.
    \end{templateoptions}

    The following commands are useful for this template:
    \begin{templateinserts}
      \iteminsert{\insertauthor} inserts a version of the author's name
      that is useful for the title page.
      \iteminsert{\insertdate} inserts the date.
      \iteminsert{\insertinstitute} inserts the institute.
      \iteminsert{\inserttitle} inserts a version of the document title
      that is useful for the title page.
      \iteminsert{\insertsubtitle} inserts a version of the document title
      that is useful for the title page.
      \iteminsert{\inserttitlegraphic} inserts the title graphic into a
      template. 
    \end{templateinserts}
  \end{element}
\end{command}

For compatibility with other classes, in |article| mode the following
command is also provided: 

\begin{command}{\maketitle}
  \beamernote
  If used inside a frame, it has the same effect as |\titlepage|. If
  used outside a frame, it has the same effect as
  |\frame{\titlepage}|; in other words, a frame is added if necessary.
\end{command}


Before you invoke the title page command, you must specify all
elements you wish to be shown. This is done using the following
commands: 

\begin{command}{\title\oarg{short title}\marg{title}}
  The \meta{short tile} is used in headlines and footlines. Inside
  the \meta{title} line breaks can be inserted using the
  double-backslash command.
  \example
\begin{verbatim}
\title{The Beamer Class}
\title[Short Version]{A Very Long Title\\Over Several Lines}
\end{verbatim}

  \articlenote
  The short form is ignored in |article| mode.
\end{command}

\begin{command}{\subtitle\oarg{short subtitle}\marg{subtitle}}
  The \meta{subshort tile} is not used by default, but is available
  via the insert |\insertshortsubtitle|. The subtitle is shown below
  the title in a smaller font.
  \example
\begin{verbatim}
\title{The Beamer Class}
\subtitle{An easily paced introduction with many examples.}
\end{verbatim}

  \articlenote
  This command causes the subtitle to be appended to the title with a
  linebreak and a |\normalsize| command issued before it. This may or
  may not be what you would like to happen.
\end{command}

\begin{command}{\author\oarg{short author names}\marg{author names}}
  The names should be separated using the
  command |\and|. In case authors have different affiliations,
  they should be suffixed by the command |\inst| with different
  parameters.
  \example|\author[Hemaspaandra et al.]{L. Hemaspaandra\inst{1} \and T. Tantau\inst{2}}|

  \articlenote
  The short form is ignored in |article| mode.
\end{command}

\begin{command}{\institute\oarg{short institute}\marg{institute}}
  If more than one institute is given, they should be separated using
  the command |\and| and they should be prefixed by the command
  |\inst| with different parameters.
  \example
\begin{verbatim}
\institute[Universities of Rochester and Berlin]{
  \inst{1}Department of Computer Science\\
  University of Rochester
  \and
  \inst{2}Fakult\"at f\"ur Elektrotechnik und Informatik\\
  Technical University of Berlin}
\end{verbatim}

  \articlenote
  The short form is ignored in |article| mode. The long form is also
  ignored, except if the document class (like |llncs|) defines it.
\end{command}

\begin{command}{\date\oarg{short date}\marg{date}}
  \example|\date{\today}| or |\date[STACS 2003]{STACS Conference, 2003}|.

  \articlenote
  The short form is ignored in |article| mode.
\end{command}

\begin{command}{\titlegraphic\marg{text}}
  The \meta{text} is shown as title graphic. Typically, a picture
  environment is used as \meta{text}.
  \example|\titlegraphic{\pgfuseimage{titlegraphic}}|

  \articlenote
  The command is  ignored in |article| mode.
\end{command}




\begin{command}{\subject\marg{text}}
  Enters the \meta{text} as the subject text in the \pdf\ document
  info. It currently has no other effect.
\end{command}

\begin{command}{\keywords\marg{text}}
  Enters the \meta{text} as keywords in the \pdf\ document
  info. It currently has no other effect.
\end{command}


By default, the |\title| and |\author| commands will also insert their
arguments into a resulting \pdf-file in the document information
fields. This may cause problems if you use complicated things like
boxes as arguments to these commands. In this case, you might wish to
switch off the automatic generation of these entries using the
following class option:

\begin{classoption}{usepdftitle=false}
  Suppresses the automatic generation of title and author entries in
  the \pdf\ document information.
\end{classoption}





\subsubsection{Adding Sections and Subsections}

You can structure your text using the commands |\section| and
|\subsection|. Unlike standard \LaTeX, these commands will not
create a heading at the position where you use them. Rather, they will
add an entry to the table of contents and also to the navigation
bars.

In order to create a line break in the table of contents (usually not
a good idea), you can use the command |\breakhere|. Note that the
standard command |\\| does not work (actually, I do not really know
why; comments would be appreciated).

\begin{command}{\section\sarg{mode specification}\oarg{short section name}\marg{section name}}
  Starts a section. No heading is created. The \meta{section name}
  is shown in the table of contents and in the navigation bars, except
  if \meta{short section name} is specified. In this case, \meta{short
    section name} is used in the navigation bars instead. If a
    \meta{mode specification} is given, the command only has an effect
    for the specified modes.
    
  \example|\section[Summary]{Summary of Main Results}|

  \articlenote
  The \meta{mode specification} allows you to provide an alternate
  section command in |article| mode. This is necessary for example if
  the \meta{short section name} is unsuitable for the table of
  contents:

  \example
\begin{verbatim}
\section<presentation>[Results]{Results on the Main Problem}
\section<article>{Results on the Main Problem}
\end{verbatim}

  \begin{element}{section in toc}\yes\yes\yes
    This template is used when a section entry is to be typeset. For
    the permissible \meta{options} see the parent template
    |table of contents|.

    The following commands are useful for this template:
    \begin{templateinserts}
      \iteminsert{\inserttocsection}
      inserts the table of contents version of the current section name.

      \iteminsert{\inserttocsectionnumber}
      inserts the number of the current section (in the table of contents).
    \end{templateinserts}
  \end{element}

  \begin{element}{section in toc shaded}\yes\yes\yes
    This template is used instead of the previous one if the section
    should be shown in a shaded way, because it is not the current
    section.  For the permissible \meta{options} see the parent
    template |table of contents|.
  \end{element}
\end{command}

\begin{command}{\section\sarg{mode specification}\declare{|*|}\marg{section name}}
  Starts a section without an entry in the table of contents. No
  heading is created, but the \meta{section name} is shown in the
  navigation bar. 
  \example|\section*{Outline}|
  \example|\section<beamer>*{Outline}|
\end{command}

\begin{command}{\subsection\sarg{mode specification}\oarg{short
  subsection name}\marg{subsection name}} 
  This command works the same way as the |\section| command.
  \example|\subsection[Applications]{Applications to the Reduction of Pollution}|

  \begin{element}{subsection in toc}\yes\yes\yes
    Like |section in toc|, only for subsection.

    In addition to the inserts for the |section in toc| template, the
    following commands are available for this template: 
    \begin{templateinserts}
      \iteminsert{\inserttocsubsection}
      inserts the table of contents version of the current subsection
      name.

      \iteminsert{\inserttocsubsectionnumber}
      inserts the number of the current subsection (in the table of contents).
    \end{templateinserts}
  \end{element}

  \begin{element}{subsection in toc shaded}\yes\yes\yes
    Like |section in toc shaded|, only for subsections.
  \end{element}
\end{command}

\begin{command}{\subsection\sarg{mode specification}\declare{|*|}\marg{subsection name}} 
  Starts a subsection without an entry in the table of contents. No
  heading is created, but the \meta{subsection name} is shown in the
  navigation bar, \emph{except} if \meta{subsection name} is empty. In
  this case, neither a table of contents entry nor a navigation bar
  entry is created, \emph{but} any frames in this ``empty'' subsection
  are shown in the navigation bar.

  \example
\begin{verbatim}
\section{Summary}

  \frame{This frame is not shown in the navigation bar}

  \subsection*{}

  \frame{This frame is shown in the navigation bar, but no subsection
    entry is shown.}

  \subsection*{A subsection}

  \frame{Normal frame, shown in navigation bar. The subsection name is
    also shown in the navigation bar, but not in the table of contents.} 
\end{verbatim}
\end{command}

Often, you may want a certain type of frame to be shown directly after
a section or subsection starts. For example, you may wish every
subsection to start with a frame showing the table of contents with
the current subsection hilighted. To facilitate this, you can use the
following two commands.


\begin{command}{\AtBeginSection\oarg{special star text}\marg{text}}
  The given text will be inserted at the beginning of every
  section. If the \meta{special star text} parameter is specified,
  this text will be used for starred sections instead. Different calls
  of this command will not ``add up'' the given texts (like the
  |\AtBeginDocument| command does), but will  overwrite any previous
  text. 
  
  \example
\begin{verbatim}
\AtBeginSection[] % Do nothing for \section*
{
  \begin{frame}<beamer>
    \frametitle{Outline}
    \tableofcontents[currentsection]
  \end{frame}
}
\end{verbatim}

  \articlenote
  This command has no effect in |article| mode.

  \lyxnote
  You have to insert this command using a \TeX-mode text.
\end{command}


\begin{command}{\AtBeginSubsection\oarg{special star text}\marg{text}}
  The given text will be inserted at the beginning of every
  subsection. If the \meta{special star text} parameter is specified,
  this text will be used for starred subsections instead. Different calls
  of this command will not ``add up'' the given texts.
  
  \example
\begin{verbatim}
\AtBeginSubsection[] % Do nothing for \subsection*
{
  \begin{frame}<beamer>
    \frametitle{Outline}
    \tableofcontents[currentsection,currentsubsection]
  \end{frame}
}
\end{verbatim}
\end{command}




\subsubsection{Adding Parts}

If you give a long talk (like a lecture), you may wish to break up
your talk into several parts. Each such part acts like a little ``talk
of its own'' with its own table of contents, its own navigation bars,
and so on. Inside one part, the sections and subsections of the other
parts are not shown at all.

To create a new part, use the |\part| command. All sections and
subsections following this command will be ``local'' to that part.
Like the |\section| and |\subsection| command, the |\part| command
does not cause any frame or special text to be produced. However,
it is often advisable for the start of a new part to use the command
|\partpage| to insert some text into a frame that ``advertises'' the
beginning of a new part. See |beamerexample3.tex| for an example.

\begin{command}{\part\sarg{mode specification}\oarg{short part name}\marg{part name}}
  Starts a part. The \meta{part name} will be shown when the
  |\partpage| command is used. The \meta{shown part name} is not shown
  anywhere by default, but it is accessible via the command
  |\insertshortpart|.
  \example
\begin{verbatim}
\begin{document}
  \frame{\titlepage}

  \section*{Outlines}
  \subsection{Part I: Review of Previous Lecture}
  \frame{
    \frametitle{Outline of Part I}
    \tableofcontents[part=1]}
  \subsection{Part II: Today's Lecture}
  \frame{
    \frametitle{Outline of Part II}
    \tableofcontents[part=2]}

  \part{Review of Previous Lecture}
  \frame{\partpage}
  \section[Previous Lecture]{Summary of the Previous Lecture}
  \subsection{Topics}
  \frame{...}
  \subsection{Learning Objectives}
  \frame{...}
  
  \part{Today's Lecture}
  \frame{\partpage}
  \section{Topic A}
  \frame{\tableofcontents[currentsection]}
  \subsection{Foo}
  \frame{...}
  \section{Topic B}
  \frame{\tableofcontents[currentsection]}
  \subsection{bar}
  \frame{...}
\end{document}
\end{verbatim}
\end{command}

\begin{command}{\partpage}
  Works like |\titlepage|, only that the current part, not the current
  presentation is ``advertised.''
  \example |\frame{\partpage}|
  
  \begin{element}{part page}\yes\yes\yes
    This template is invoked when the |\partpage| command is used.

    \begin{templateoptions}
      \itemoption{default}{\oarg{orientation}}
      The part page is typeset showing the current part number and,
      below, the current part title. The \beamer-color and -font
      |part page| are used, including the background color of
      |part page|. As for the |title page| template, the
      \meta{orientation} option is passed on the |beamercolorbox|.
    \end{templateoptions}

    The following commands are useful for this template:
    \begin{templateinserts}
      \iteminsert{\insertpart}
      inserts the current part name.
      \iteminsert{\insertpartnumber}
      inserts the current part number as an Arabic number into a template.
      \iteminsert{\insertpartromannumber}
      inserts the current part number as a Roman number into a template.
    \end{templateinserts}
  \end{element}
\end{command}

\begin{command}{\AtBeginPart\marg{text}}
  The given text will be inserted at the beginning of every
  part.
  
  \example
\begin{verbatim}
\AtBeginPart{\frame{\partpage}}
\end{verbatim}
\end{command}


\subsubsection{Splitting a Course Into Lectures}

When using \beamer\ with the |article| mode, you may wish to have the
lecture notes of a whole course reside in one file. In this case, only
a few frames are actually part of any particular lecture.

The |\lecture| command makes it easy to select only a certain set of
frames from a file to be presented. This command takes (among other
things) a label name. If you say |\includeonlylecture| with this label
name, then only the frames following the corresponding |\lecture|
command are shown. The frames following other |\lecture| commands are
suppressed. 

By default, the |\lecture| command has no other effect. It does not
create any frames or introduce entries in the table of
contents. However, you can use |\AtBeginLecture| to have \beamer\
insert, say, a title page at the beginning of (each) lecture.

\begin{command}{\lecture\oarg{short lecture name}\marg{lecture
  name}\marg{lecture label}} 
  Starts a lecture. The \meta{lecture name} will be available via the
  |\insertlecture| command. The \meta{short lecture name} is available
  via the |\insertshortlecture| command.
  
  \example
\begin{verbatim}
\begin{document}
\lecture{Vector Spaces}{week 1}

\section{Introduction}
...
\section{Summary}

\lecture{Scalar Products}{week 2}

\section{Introduction}
...
\section{Summary}

\end{document}
\end{verbatim}

  \articlenote
  This command has no effect in |article| mode.
\end{command}

\begin{command}{\includeonlylecture\meta{lecture label}}
  Causes all |\frame|, |frame|, |\section|, |\subsection|, and |\part|
  commands following a |\lecture| command to be suppressed, except if the
  lecture's label matches the \meta{lecture label}. Frames before any
  |\lecture| commands are always included. This command should be
  given in the preamble.

  \example |\includeonlylecture{week 1}|

  \articlenote
  This command has no effect in |article| mode.
\end{command}

\begin{command}{\AtBeginLecture\marg{text}}
  The given text will be inserted at the beginning of every
  lecture.
  
  \example
\begin{verbatim}
\AtBeginLecture{\frame{\Large Today's Lecture: \insertlecture}}
\end{verbatim}

  \articlenote
  This command has no effect in |article| mode.
\end{command}


\subsubsection{Adding a Table of Contents}

You can create a table of contents using the command
|\tableofcontents|. Unlike the normal \LaTeX\ table of contents
command, this command takes an optional parameter in square brackets
that can be used to create certain special effects.

\begin{command}{\tableofcontents\oarg{comma-separated option list}}
  Inserts a table of contents into the current frame. To change how
  the table of contents is typeset, you need to modify the appropriate
  templates, see Section~\ref{section-toc-templates}. 
  \example
\begin{verbatim}
\section*{Outline}
\frame{\tableofcontents}

\section{Introduction}
\frame{\tableofcontents[currentsection]}
\subsection{Why?}
\frame{...}
\frame{...}
\subsection{Where?}
\frame{...}

\section{Results}
\frame{\tableofcontents[currentsection]}
\subsection{Because}
\frame{...}
\subsection{Here}
\frame{...}
\end{verbatim}

  The following options can be given:
  \begin{itemize}
  \item
    \declare{|currentsection|} causes all sections but the current to
    be shown in a semi-transparent way. Also, all subsections but
    those in the current section are shown in the semi-transparent
    way. This command is a shorthand for specifying the following
    options: 
\begin{verbatim}
sectionstyle=show/shaded,subsectionstyle=show/show/shaded
\end{verbatim}
  \item
    \declare{|currentsubsection|} causes all subsections but the
    current subsection in the current section to be shown in a
    semi-transparent way. This command is a shorthand for specifying
    the option |subsectionstyle=show/shaded|.
  \item
    \declare{|firstsection=|\meta{section number}} specifies which
    section should be numbered as section~``1.''  This is useful if
    you have a first section (like an overview section) that should
    not receive a number. Section numbers are not shown by default. To
    show them, you must install a different table of contents
    templates.
  \item
    \declare{|hideallsubsections|} causes all subsections to be
    hidded. This command is a shorthand for specifying
    the option |subsectionstyle=hide|.
  \item
    \declare{|hideothersubsections|} causes the subsections of
    sections other than the current one to be hidded. This command is
    a shorthand for specifying the option |subsectionstyle=show/show/hide|.
  \item
    \declare{|part=|\meta{part number}} causes the table of contents
    of part \meta{part number} to be shown, instead of the table of
    contents of the current part (which is the default). This option
    can be combined with the other options, although combining it with
    the |current| option obviously makes no sense.
  \item
    \declare{|pausesections|} causes a |\pause| command to
    be issued before each section. This is useful if you wish to show
    the table of contents in an incremental way.
  \item
    \declare{|pausesubsections|} causes a |\pause| command to
    be issued before each subsection.
  \item
    \declare{|sections=|\marg{overlay specification}} causes only the
    sections mentioned in the \meta{overlay specification} to be
    shown. For example, \verb/sections={<2-4| handout:0>}/ causes only the second,
    third, and fourth section to be shown in the normal version,
    nothing to be shown in the handout version, and everything to be
    shown in all other versions. For convenience, if you omit the
    pointed brackets, the specification is assumed to apply to all
    versions. Thus |sections={2-4}| causes sections two, three, and
    four to be shown in all versions.
  \item
    \declare{|sectionstyle=|\meta{style for current
      section}|/|\meta{style for other sections}} specifies how
    sections should be displayed. Allowed \meta{styles} are |show|,
    |shaded|, and |hide|. The first will show the section title
    normally, the second will show it in a semi-transparent way, and
    the third will completely suppress it. You can
    also omit the second style, in which case the first is used for
    all sections (this is not really useful). 
  \item
    \declare{|subsectionstyle=|\meta{style for current
      subsection}|/|\meta{style for other subsections in current
      section}|/|\\\meta{style for subsections in other sections}}
    specifies how subsections should be displayed. The same styles as
    for the |sectionstyle| option may be given. You can omit the last
    style, in which case the second also applies to the last, and you can
    omit the last two, in which case the first applies to all.
    \example |subsectionstyle=shaded| causes all subsections to be
    shaded.
    \example |subsectionstyle=hide| causes all subsections to be
    hidden.
    \example |subsectionstyle=show/shaded| causes all subsections but the
    current subsection in the current section to be shown in a
    semi-transparent way.
    \example |subsectionstyle=show/show/hide| causes all
    subsections outside the current section to be suppressed.
    \example |subsectionstyle=show/shaded/hide| causes all
    subsections outside the current section to be suppressed and only
    the current subsection in the current section to be hilighted.
  \end{itemize}
  The last examples are useful if you do not wish to show
  too many details when presenting the talk outline.

  \articlenote
  The options are ignored in |article| mode.

  \lyxnote
  You can give options to the |\tableofcontents| command by 
  inserting a \TeX-mode text with the options in square brackets
  directly after the table of contents.

  \begin{element}{sections/subsections in toc}\yes\no\no
    This template is a parent template, whose children are
    |section in toc| and |subsection in toc|. This means that if you
    use the |\setbeamertemplate| command on this template, the command
    is instead called for both of these children (with the same arguments).

    \begin{templateoptions}
      \itemoption{default}{}
      In the default setting, the sections and subsections are typeset
      using the fonts and colors |section in toc| and
      |subsection in toc|, though the background colors are ignored. The
      subsections are indented.
      \itemoption{sections numbered}{}
      Similar to the default setting, but the section numbers are also
      shown. The subsections are not numbered.
      \itemoption{subsections numbered}{}
      This time, the subsections are numbered, but not the
      sections. Nevertheless, since the subsections are ``fully
      numbered'' as in ``1.2'' or ``3.2,'' if every section has at least
      one subsection, the section numbered will not really be missed.
      \itemoption{circle}{}
      Draws little circles before the sections. Inside
      the circles the section number is shown. The \beamer-font and
      color |section number projected| is used for typesetting the
      circles, that is, the circle gets the background color and the
      text inside the circle the foreground color.
      \itemoption{square}{}
      Similar to the |circle| option, except that small squares are
      used instead of circles. Small, unnumbered squares are shown in
      front of the subsections.
      \itemoption{ball}{}
      Like |square|, the only difference being the balls are used
      instead of squares.
      \itemoption{ball unnumbered}{}
      Similar to |ball|, except that no numbering is used. This option
      makes the table of contents look more like an |itemize|.
    \end{templateoptions}

    If none of the above options suits you, you have to change the
    templates |section in toc| and |subsection in toc| directly.
  \end{element}

  \begin{element}{sections/subsections in toc shaded}\yes\no\no
    A parent template with children
    |section in toc shaded| and |subsection in toc shaded|. These two
    templates are used to render section and subsection entries when
    they are currently shaded; like all non-current subsections in
    |\tableofcontents[currentsubsection]|. 

    \begin{templateoptions}
      \itemoption{default}{\oarg{opaquness}}
      In the default setting, the templates |section in toc shaded| and
      |subsection in toc shaded| just show whatever the nonshaded
      versions of these templates show, but only \meta{opaquness}\%
      opaque. The default is 20\%.

      \example |\setbeamertemplate{table of contents shaded}[default][50]|
      makes dimmed entries 50\% transparent. 
    \end{templateoptions}
  \end{element}
\end{command}





\subsubsection{Adding a Bibliography}

You can use the bibliography environment and the |\cite| commands
of \LaTeX\ in a \beamer\ presentation. You will typically have to
typeset your bibliography items partly ``by hand.'' Nevertheless, you
\emph{can} use |bibtex| to create a ``first approximation'' of the
bibliography. Copy the content of the file |main.bbl| into your
presentation. If you are not familiar with |bibtex|, you may wish
to consult its documentation. It is a  powerful tool for
creating high-quality citations.

Using |bibtex| or your editor, place your bibliographic
references in the environment |thebibliography|. This
(standard \LaTeX) environment takes one parameter, which should be the
longest |\bibitem| label in the following list of bibliographic
entries.

\begin{environment}{{thebibliography}\marg{longest label text}}
  Inserts a bibliography into the current frame. The \meta{longest
    label text} is used to determine the indent of the list. However,
  several templates for the typesetting of the bibliography (see
  Section~\ref{section-bib-templates}) ignore this 
  parameter since they replace the references by a symbol.

  Inside the environment, use a (standard \LaTeX) |\bibitem| command
  for each reference item. Inside each item, use a (standard \LaTeX)
  |\newblock| command to separate the authors's names, the title, the
  book/journal reference, and any notes. Each of these commands may
  introduce a new line or color or other formatting, as specified by
  the template for bibliographies.

  The environment must be placed inside a frame. If the bibliography
  does not fit on one frame, you should 
  split it (create a new frame and a second |thebibliography|
  environment) or use the |allowframebreaks| option. Even better, you
  should reconsider whether it is a good idea to present so many
  references. 
  \example
\begin{verbatim}
\begin{frame}
  \frametitle{For Further Reading}

  \begin{thebibliography}{Dijkstra, 1982}
  \bibitem[Solomaa, 1973]{Solomaa1973}
    A.~Salomaa.
    \newblock {\em Formal Languages}.
    \newblock Academic Press, 1973.

  \bibitem[Dijkstra, 1982]{Dijkstra1982}
    E.~Dijkstra.
    \newblock Smoothsort, an alternative for sorting in situ.
    \newblock {\em Science of Computer Programming}, 1(3):223--233, 1982.
  \end{thebibliography}
\end{frame}
\end{verbatim}
\end{environment}

\begin{command}{\bibitem\sarg{overlay specification}%
    \oarg{citation text}\marg{label name}}
  The \meta{citation text} is inserted into the text when the item is
  cited using |\cite{|\meta{label name}|}| in the main presentation
  text. For a \beamer\ presentation, this should usually be as long as
  possible.  

  Use |\newblock| commands to separate the authors's names, the title, the
  book/journal reference, and any notes. If the \meta{overlay specification}
  is present, the entry will only be shown on the
  specified slides.
  \example
\begin{verbatim}
\bibitem[Dijkstra, 1982]{Dijkstra1982}
  E.~Dijkstra.
  \newblock Smoothsort, an alternative for sorting in situ.
  \newblock {\em Science of Computer Programming}, 1(3):223--233, 1982.
\end{verbatim}
\end{command}

Unlike normal \LaTeX, the default template for the
bibliography does not repeat the citation text (like ``[Dijkstra,
1982]'') before each item in the bibliography. Instead, a cute, small
article symbol is drawn. The rationale is that the audience will not be
able to remember any abbreviated citation texts till the end of the
talk. If you really insist on using abbreviations, you can use the
command |\beamertemplatetextbibitems| to restore the default
behavior, see also Section~\ref{section-bib-templates}.




\subsubsection{Adding an Appendix}

You can add an appendix to your talk by using the |\appendix|
command. You should put frames and perhaps whole subsections into the
appendix that you do not intend to show during your presentation, but
which might be useful to answer a question. The |\appendix| command
essentially just starts a new part named |\appendixname|. However, it
also sets up certain hyperlinks. 
Like other parts, the appendix is kept separate of your actual
talk.

\begin{command}{\appendix\sarg{mode specification}}
  Starts the appendix in the specified modes. All frames, all
  |\subsection| commands, and all |\section| commands used after this
  command will not be shown as part of the normal navigation bars.
  \example
\begin{verbatim}
\begin{document}
\frame{\titlepage}
\section*{Outline}
\frame{\tableofcontents}
\section{Main Text}
\frame{Some text}
\section*{Summary}
\frame{Summary text}

\appendix
\section{\appendixname}
\frame{\tableofcontents}
\subsection{Additional material}
\frame{Details}
\frame{Text omitted in main talk.}
\subsection{Even more additional material}
\frame{More details}
\end{document}
\end{verbatim}
\end{command}




\subsection{Creating the Interactive Global Structure}

\label{section-nonlinear}


\subsubsection{Adding Hyperlinks and Buttons}

To create anticipated nonlinear jumps in your talk structure, you
can add hyperlinks to your presentation. A hyperlink is a text
(usually rendered as a button) that, when you click on it, jumps the
presentation to some other slide. Creating such a button is a
three-step process: 
\begin{enumerate}
\item
  You specify a target using the command |\hypertarget| or (easier)
  the command |\label|. In some cases, see below, this step may be
  skipped. 
\item
  You render the button using |\beamerbutton| or a similar
  command. This will \emph{render} the button, but clicking it will
  not yet have any effect. 
\item
  You put the button inside a |\hyperlink| command. Now clicking it
  will jump to the target of the link.  
\end{enumerate}

\begin{command}{\hypertarget\sarg{overlay specification}%
    \marg{target name}\marg{text}}
  If the \meta{overlay specification} is present, the \meta{text} is
  the target for hyper jumps to \meta{target name} only on the
  specified slide. On all other slides, the text is shown
  normally. Note that you \emph{must} add an overlay specification to
  the |\hypertarget| command whenever you use it on frames that have
  multiple slides (otherwise |pdflatex| rightfully complains
  that you have defined the same target on different slides).
  \example
\begin{verbatim}
\begin{frame}
  \begin{itemize}
  \item<1-> First item.
  \item<2-> Second item.
  \item<3-> Third item.
  \end{itemize}

  \hyperlink{jumptosecond}{\beamergotobutton{Jump to second slide}}
  \hypertarget<2>{jumptosecond}{}
\end{frame}
\end{verbatim}

  \articlenote
  You must say |\usepackage{hyperref}| in your preamble to use this
  command in |article| mode.
\end{command}

The |\label| command creates a hypertarget as a side-effect and the
|label=|\meta{name} option of the |\frame| command creates a label
named \meta{name}|<|\meta{slide number}|>| for each slide of the frame
as a side-effect. Thus the above example could be written more easily
as: 
\begin{verbatim}
\begin{frame}[label=threeitems]
  \begin{itemize}
  \item<1-> First item.
  \item<2-> Second item.
  \item<3-> Third item.
  \end{itemize}

  \hyperlink{threeitems<2>}{\beamergotobutton{Jump to second slide}}
\end{frame}
\end{verbatim}



The following commands can be used to specify in an abstract way what
a button will be used for. How exactly these buttons are rendered is
governed by a template, see Section~\ref{section-navigation-buttons}.

\begin{command}{\beamerbutton\marg{button text}}
  Draws a button with the given \meta{button text}.
  \example |\hyperlink{somewhere}{\beamerbutton{Go somewhere}}|

  \articlenote
  This command (and the following) just insert their argument in
  |article| mode.
\end{command}

\begin{command}{\beamergotobutton\marg{button text}}
  Draws a button with the given \meta{button text}. Before the text, a
  small symbol (usually a right-pointing arrow) is inserted that
  indicates that pressing this button will jump to another ``area'' of
  the presentation.

  \example |\hyperlink{detour}{\beamergotobutton{Go to detour}}|
\end{command}

\begin{command}{\beamerskipbutton\marg{button text}}
  The symbol drawn for this button is usually a double right
  arrow. Use this button if pressing it will skip over a
  well-defined part of your talk.

  \example
\begin{verbatim}
\frame{
  \begin{theorem}
    ...
  \end{theorem}

  \begin{overprint}
  \onslide<1>
    \hfill\hyperlinkframestartnext{\beamerskipbutton{Skip proof}}
  \onslide<2>
    \begin{proof}
      ...
    \end{proof}
  \end{overprint}
}
\end{verbatim}
\end{command}

\begin{command}{\beamerreturnbutton\marg{button text}}
  The symbol drawn for this button is usually a left-pointing
  arrow. Use this button if pressing it will return from a detour. 

  \example
\begin{verbatim}
\frame<1>[label=mytheorem]
{
  \begin{theorem}
    ...
  \end{theorem}

  \begin{overprint}
  \onslide<1>
    \hfill\hyperlink{mytheorem<2>}{\beamergotobutton{Go to proof details}}
  \onslide<2>
    \begin{proof}
      ...
    \end{proof}
    \hfill\hyperlink{mytheorem<1>}{\beamerreturnbutton{Return}}
  \end{overprint}
}
\appendix
\againframe<2>{mytheorem}
\end{verbatim}
\end{command}

To make a button ``clickable'' you must place it in a command like
|\hyperlink|. The command |\hyperlink| is a standard command of the
|hyperref| package. The \beamer\ class defines a whole bunch of other
hyperlink commands that you can also use.

\begin{command}{\hyperlink\sarg{overlay specification}\marg{target
      name}\marg{link text}\sarg{overlay specification}}
  Only one \meta{overlay specification} may be given.
  The \meta{link text} is typeset in the usual way. If you click
  anywhere on this text, you will jump to the slide on which the
  |\hypertarget| command was used with the parameter \meta{target
    name}. If an \meta{overlay specification} is present, the
    hyperlink (including the \meta{link text}) is completely
    suppressed on the non-specified slides.
\end{command}

The following commands have a predefined target; otherwise they behave
exactly like |\hyperlink|. In particular, they all also accept an
overlay specification and they also accept it at the end, rather than
at the beginning.

\begin{command}{\hyperlinkslideprev\sarg{overlay specification}\marg{link text}}
  Clicking the text jumps one slide back.
\end{command}

\begin{command}{\hyperlinkslidenext\sarg{overlay specification}\marg{link text}}
  Clicking the text jumps one slide forward.
\end{command}
  
\begin{command}{\hyperlinkframestart\sarg{overlay specification}\marg{link text}}
  Clicking the text jumps to the first slide of the current frame.
\end{command}

\begin{command}{\hyperlinkframeend\sarg{overlay specification}\marg{link text}}
  Clicking the text jumps to the last slide of the current frame.
\end{command}

\begin{command}{\hyperlinkframestartnext\sarg{overlay specification}\marg{link text}}
  Clicking the text jumps to the first slide of the next frame.
\end{command}

\begin{command}{\hyperlinkframeendprev\sarg{overlay specification}\marg{link text}}
  Clicking the text jumps to the last slide of the previous frame.
\end{command}

The previous four command exist also with ``|frame|'' replaced by
``|subsection|'' everywhere, and also again with  ``|frame|'' replaced
by ``|section|''.

\begin{command}{\hyperlinkpresentationstart\sarg{overlay specification}\marg{link text}}
  Clicking the text jumps to the first slide of the presentation.
\end{command}

\begin{command}{\hyperlinkpresentationend\sarg{overlay specification}\marg{link text}}
  Clicking the text jumps to the last slide of the presentation. This
  \emph{excludes} the appendix.
\end{command}

\begin{command}{\hyperlinkappendixstart\sarg{overlay specification}\marg{link text}}
  Clicking the text jumps to the first slide of the appendix. If there
  is no appendix, this will jump to the last slide of the document.
\end{command}

\begin{command}{\hyperlinkappendixend\sarg{overlay specification}\marg{link text}}
  Clicking the text jumps to the last slide of the appendix.
\end{command}

\begin{command}{\hyperlinkdocumentstart\sarg{overlay specification}\marg{link text}}
  Clicking the text jumps to the first slide of the presentation.
\end{command}

\begin{command}{\hyperlinkdocumentend\sarg{overlay specification}\marg{link text}}
  Clicking the text jumps to the last slide of the presentation or, if
  an appendix is present, to the last slide of the appendix.
\end{command}




\subsubsection{Repeating a Frame at a Later Point}

Sometimes you may wish some slides of a frame to be shown in your main
talk, but wish some ``supplementary'' slides of the frame to be shown
only in the the appendix. In this case, the |\againframe| commands is
useful. 


\begin{command}{\againframe\sarg{overlay
      specification}\opt{|[<|\meta{default overlay specification}|>]|}\oarg{options}\marg{name}}
  \beamernote
  Resumes a frame that was previously created using |\frame|
  with the option |label=|\meta{name}. You must have used this option,
  just placing a label inside a frame ``by hand'' is not enough. You
  can use this command to ``continue'' a frame that has been
  interrupted by another frame. The effect of this command is to call
  the |\frame| command with the given \meta{overlay specification},
  \meta{default overlay specification} (if present), and
  \meta{options} (if present) and with the original frame's contents. 

  \example
\begin{verbatim}
\frame<1-2>[label=myframe]
{
  \begin{itemize}
  \item<alert@1> First subject.
  \item<alert@2> Second subject.
  \item<alert@3> Third subject.
  \end{itemize}
}

\frame
{
  Some stuff explaining more on the second matter.
}

\againframe<3>{myframe}
\end{verbatim}
  The effect of the above code is to create four slides. In the first
  two, the items 1 and~2 are hilighted. The third slide contains the
  text ``Some stuff explaining more on the second matter.'' The fourth
  slide is identical to the first two slides, except that the third
  point is now hilighted.

  \example
\begin{verbatim}
\frame<1>[label=Cantor]
{
  \frametitle{Main Theorem}

  \begin{Theorem}
    $\alpha < 2^\alpha$ for all ordinals~$\alpha$.
  \end{Theorem}

  \begin{overprint}
  \onslide<1>
    \hyperlink{Cantor<2>}{\beamergotobutton{Proof details}}

  \onslide<2->
    % this is only shown in the appendix, where this frame is resumed.
    \begin{proof}
      As shown by Cantor, ...
    \end{proof}

    \hfill\hyperlink{Cantor<1>}{\beamerreturnbutton{Return}}
  \end{overprint}
}

...
\appendix

\againframe<2>{Cantor}
\end{verbatim}
  In this example, the proof details are deferred to a slide in the
  appendix. Hyperlinks are setup, so that one can jump to the proof
  and go back.

  \articlenote
  This command is ignored in |article| mode.

  \lyxnote
  Use the style ``AgainFrame'' to insert an |\againframe| command. The
  \meta{label name} is the text on following the style name
  and is \emph{not} put in \TeX-mode. However, an overlay specification
  must be given in \TeX-mode and it must precede the label name.
\end{command}



\subsubsection{Adding Anticipated Zooming}

\label{section-zooming}


Anticipated zooming is necessary when you have a very complicated
graphic that you are not willing to simplify since, indeed, all the
complex details merit an explanation. In this case, use the command
|\framezoom|. It allows you to specify that clicking on a certain area
of a frame should zoom out this area. You can then explain the
details. Clicking on the zoomed out picture will take you back to the
original one. 

\begin{command}{\framezoom\ssarg{button overlay
      specification}\ssarg{zoomed overlay
      specification}\oarg{options}\\|(|\meta{upper left x}|,|\meta{upper
      left y}|)(|\meta{zoom area width}|,|\meta{zoom area depth}|)|}
  This command should be given somewhere at the beginning of a
  frame. When given, two different things will happen, depending on
  whether the \meta{button overlay specification} applies to the
  current slide of the frame or whether the \meta{zoomed overlay
    specification} applies. These overlay specifications should not
  overlap.

  If the \meta{button overlay specification} applies, a clickable are
  is created inside the frame. The size of this area is given by
  \meta{zoom area width} and \meta{zoom area depth}, which are two
  normal \TeX\ dimensions (like |1cm| or |20pt|). The upper left
  corner of this area is given by \meta{upper left x} and \meta{upper
  left y}, which are also \TeX\ dimensions. They are measures
  \emph{relative to the place where the first normal text of a
    frame would go}. Thus, the location |(0pt,0pt)| is at the
  beginning of the normal text (which excludes the headline and also
  the frame title).

  By default, the button is clickable, but it will not be indicated in
  any special way. You can draw a border around the button by using
  the following \meta{option}:
  \begin{itemize}
  \item \declare{|border|}\opt{|=|\meta{width in pixels}} will draw
    a border around the specified button area. The default width is 1
    pixel. The color of this  button is the |linkbordercolor| of
    |hyperref|. \beamer\ sets this color to a 50\% gray by default. To
    change this, you can use the command
    |\hypersetup{linkbordercolor={|\meta{red}| |\meta{green}| |\meta{blue}|}}|, 
    where \meta{red}, \meta{green}, and \meta{blue} are values between
    0 and 1.
  \end{itemize}

  When you press the button created in this way, the viewer
  application will hyperjump to the first of the frames specified by
  the \meta{zoomed overlay specification}. For the slides to which
  this overlay specification applies, the following happens:

  The exact same area as the one specified before is ``zoomed out'' to
  fill the whole normal text area of the frame. Everything else,
  including the sidebars, the headlines and footlines, and even the
  frame title retain their normal size. The zooming is performed in
  such a way that the whole specified area is completely shown. The
  aspect ratio is kept correct and the zoomed area will possibly show
  more than just the specified area if the aspect ratio of this area
  and the aspect ratio of the available text area do not agree.

  Behind the whole text area (which contains the zoomed area) a big
  invisible ``Back'' button is put. Thus clicking anywhere on the text
  area will jump back to the original (unzoomed) picture.

  You can specify several zoom areas for a single frame. In this case,
  you should specify different \meta{zoomed overlay specification},
  but you can specify the same \meta{button overlay
  specification}. You cannot nest zoomings in the sense that you
  cannot have a zoom button on a slide that is in some \meta{zoomed
  overlay specification}. However, you can have overlapping and even
  nested \meta{button overlay specification}. When clicking on an area
  that belongs to several buttons, the one given last will ``win'' (it
  should hence be the smallest one).

  If you do not wish to have the frame title shown on a zoomed slide,
  you can add an overlay specification to the |\frametitle| command
  that simply suppresses the title for the slide. Also, by using the
  |plain| option, you can have the zoomed slide fill the whole page. 

  \example A simple case
\begin{verbatim}
\begin{frame}
  \frametitle{A Complicated Picture}

  \framezoom<1><2>(0cm,0cm)(2cm,1.5cm)
  \framezoom<1><3>(1cm,3cm)(2cm,1.5cm)
  \framezoom<1><4>(3cm,2cm)(3cm,2cm)

  \pgfimage[height=8cm]{complicatedimagefilename}
\end{frame}
\end{verbatim}

  \example A more complicate case in which the zoomed parts completely
  fill the frames.
\begin{verbatim}
\begin{frame}<1>[label=zooms]
  \frametitle<1>{A Complicated Picture}

  \framezoom<1><2>[border](0cm,0cm)(2cm,1.5cm)
  \framezoom<1><3>[border](1cm,3cm)(2cm,1.5cm)
  \framezoom<1><4>[border](3cm,2cm)(3cm,2cm)

  \pgfimage[height=8cm]{complicatedimagefilename}
\end{frame}
\againframe<2->[plain]{zooms}
\end{verbatim}
\end{command}



\subsubsection{Using the Navigation Bars}
\label{section-navigation-bars}

Navigation bars and symbols are two independent concepts that can be
used to navigate through a presentation. They are created
automatically. Most themes that come along with the \beamer\ class
show some kind of navigation bar during your talk. Although these
navigation bars take up quite a bit of space, they are often useful
for two reasons: 

\begin{itemize}
\item
  They provide the audience with a visual feedback of how much of your
  talk you have covered and what is yet to come. Without such
  feedback, an audience will often puzzle whether something you are
  currently introducing will be explained in more detail later on or
  not.
\item
  You can click on all parts of the navigation bar. This will directly
  ``jump'' you to the part you have clicked on. This is particularly
  useful to skip certain parts of your talk and during a ``question
  session,'' when you wish to jump back to a particular frame someone
  has asked about.
\end{itemize}

Some navigation bars can be ``compressed'' using the following option:

\begin{classoption}{compress}
  Tries to make all navigation bars as small as possible. For example,
  all small frame representations in the navigation bars for a single
  section are shown alongside each other. Normally, the representations
  for different subsections are shown in different lines. Furthermore,
  section and subsection navigations are compressed into one line.
\end{classoption}

When you click on one of the icons representing a frame in a
navigation bar (by default this is icon is a small circle), the
following happens: 
\begin{itemize}
\item
  If you click on (the icon of) any frame other than the current frame, the
  presentation will jump to the first slide of the frame you clicked
  on.
\item
  If you click on the current frame and you are not on the last slide
  of this frame, you will jump to the last slide of the frame.
\item
  If you click on the current frame and you are on the last slide, you
  will jump to the first slide of the frame.
\end{itemize}
By the above rules you can:
\begin{itemize}
\item
  Jump to the beginning of a frame from somewhere else by clicking on
  it once.  
\item
  Jump to the end of a frame from somewhere else by clicking on it
  twice.
\item
  Skip the rest of the current frame by clicking on it once.
\end{itemize}

I also tried making a jump to an already-visited frame jump
automatically to the last slide of this frame. However, this turned
out to be more confusing than helpful. With the current implementation
a double-click always brings you to the end of a slide, regardless
from where you ``come.''

By clicking on a section or subsection in the navigation bar, you will
jump to that section. Clicking on a section is particularly useful if
the section starts with a |\tableofcontents[currentsection]|, since you
can use it to jump to the different subsections.

By clicking on the document title in a navigation bar (not all themes
show it), you will jump to the first slide of your presentation
(usually the title page) \emph{except} if you are already at the first
slide. On the first slide, clicking on the document title will jump to
the end of the presentation, if there is one. Thus by \emph{double}
clicking the document title in a navigation bar, you can jump to the end.



\subsubsection{Using the Navigation Symbols}
\label{section-navigation-symbols}

Navigation symbols are small icons that are shown on every slide
by default. The following symbols are shown: 
\begin{enumerate}
\item
  A slide icon, which is depicted as  a single rectangle. To the left and
  right of this symbol, a left and right arrow are shown.
\item
  A frame icon, which is depicted as three slide icons ``stacked on top of
  each other''. This symbol is framed by arrows.
\item
  A subsection icon, which is depicted as a highlighted subsection
  entry in a table of contents. This  symbols is framed by arrows.
\item
  A section icon, which is depicted as a highlighted section entry
  (together with all subsections) in a table of contents. This symbol
  is framed by arrows.
\item
  A presentation icon, which is depicted as a completely highlighted
  table of contents.
\item
  An appendix icon, which is depicted as a completely highlighted
  table of contents consisting of only one section. (This icon is only
  shown if there is an appendix.)
\item
  Back and forward icons, depicted as circular arrows.
\item
  A ``search'' or ``find'' icon, depicted as a detective's
  magnifying glass.
\end{enumerate}

Clicking on the left arrow next to an icon always jumps to (the
last slide of) the previous slide, frame, subsection, or
section. Clicking on the right arrow next to an icon always jump to
(the first slide of) the next slide, frame, subsection, or section. 

Clicking \emph{on} any of these icons has different effects:
\begin{enumerate}
\item
  If supported by the viewer application, clicking on a slide icon
  pops up a window that allows you to enter a slide number to which
  you wish to jump.
\item
  Clicking on the left side of a frame icon will jump to the first
  slide of the frame, clicking on the right side will jump to the last
  slide of the frame (this can be useful for skipping overlays).
\item
  Clicking on the left side of a subsection icon will jump to the
  first slide of the subsection, clicking on the right side will jump
  to the last slide of the subsection.
\item
  Clicking on the left side of a section icon will jump to the
  first slide of the section, clicking on the right side will jump
  to the last slide of the section.
\item
  Clicking on the left side of the presentation icon will jump to the
  first slide, clicking on the right side will jump to the last slide
  of the presentation. However, this does \emph{not} include the
  appendix. 
\item
  Clicking on the left side of the appendix icon will jump to the
  first slide of the appendix, clicking on the right side will jump to
  the last slide of the appendix.
\item
  If supported by the viewer application, clicking on the back and
  forward symbols jumps to the previously visited slides.
\item
  If supported by the viewer application, clicking on the search icon
  pops up a window that allows you to enter a search string. If found,
  the viewer application will jump to this string.
\end{enumerate}

You can reduce the number of icons that are shown or their layout by
adjusting the navigation symbols template, see
Section~\ref{section-navigation-symbols-template}. 







\subsection{Creating the Local Structure}

\LaTeX\ provides different commands for structuring text ``locally,''
for example, via the |itemize| environment. These environments
are also available in the \beamer\ class, although their appearance has
been slightly changed. Furthermore, the \beamer\ class also defines
some new commands and environments, see below, that may help you to
structure your text.


\subsubsection{Itemizations, Enumerations, and Descriptions}

\label{section-enumerate}

There are three predefined environments for creating lists, namely
|enumerate|, |itemize|, and |description|. The first
two can be nested to depth three, but nesting them to this depth
creates totally unreadable slides.

The |\item| command is overlay-specification-aware. If an overlay
specification is provided, the item will only be shown on the
specified slides, see the following example. If the |\item|
command is to take an optional argument and an overlay specification,
the overlay specification can either come first as in |\item<1>[Cat]|
or come last as in |\item[Cat]<1>|.

\begin{verbatim}
\begin{frame}
  There are three important points:
  \begin{enumerate}
  \item<1-> A first one,
  \item<2-> a second one with a bunch of subpoints,
    \begin{itemize}
    \item first subpoint. (Only shown from second slide on!).
    \item<3-> second subpoint added on third slide.
    \item<4-> third subpoint added on fourth slide.
    \end{itemize}
  \item<5-> and a third one.
  \end{enumerate}
\end{frame}
\end{verbatim}


\begin{environment}{{itemize}\opt{|[<|\meta{default overlay specification}|>]|}}
  Used to display a list of items that do not have a special
  ordering. Inside the environment, use an |\item| command for
  each topic.
  
  If the optional parameter \meta{default overlay specification} is
  given, in every occurrence of an |\item| command that does not have
  an overlay specification attached to it, the \meta{default overlay
    specification} is used. By setting this specification to be an
  incremental overlay specification, see
  Section~\ref{section-incremental}, you can implement, for example, a
  step-wise uncovering of the items. The \meta{default overlay
    specification} is inherited by subenvironments. Naturally, in a
  subenvironment you can reset it locally by setting it to |<1->|.
  \example
\begin{verbatim}
\begin{itemize}
\item This is important.
\item This is also important.
\end{itemize}
\end{verbatim}
  
  \example
\begin{verbatim}
\begin{itemize}[<+->]
\item This is shown from the first slide on.
\item This is shown from the second slide on.
\item This is shown from the third slide on.
\item<1-> This is shown from the first slide on.
\item This is shown from the fourth slide on.    
\end{itemize}
\end{verbatim}
  
  \example
\begin{verbatim}
\begin{itemize}[<+-| alert@+>]
\item This is shown from the first slide on and alerted on the first slide.
\item This is shown from the second slide on and alerted on the second slide.
\item This is shown from the third slide on and alerted on the third slide.
\end{itemize}
\end{verbatim}
  
  \example
\begin{verbatim}
\newenvironment{mystepwiseitemize}{\begin{itemize}[<+-| alert@+>]}{\end{itemize}}
\end{verbatim}

  \lyxnote
  Unfortunately, currently you cannot specify optional arguments with
  the |itemize| environment. You can, however, use the command
  |\beamerdefaultoverlayspecification| before the environment to get
  the desired effect.

  The appearance of an |itemize| list is governed by several
  templates. The first template concerns the way the little marker
  introducing each item is typeset:  
  \begin{element}{itemize items}\semiyes\no\no
    This template is a parent template, whose children are
    |itemize item|, |itemize subitem|, and |itemize subsubitem|. This
    means that if you use the |\setbeamertemplate| command on this
    template, the command is instead called for all of these children
    (with the same arguments). 

    \begin{templateoptions}
      \itemoption{default}{}
      The default item marker is a small triangle having the
      foreground color |itemize item| (or, for subitems, |itemize subitem|
      etc.). Note that these colors will automatically change under
      certain circumstances such as inside an example block or inside
      an |alertenv| environment. 
      \itemoption{circle}{}
      Uses little circles (or dots) as item markers. 
      \itemoption{square}{}
      Uses little squares as item markers.
      \itemoption{ball}{}
      Uses little balls as item markers.
    \end{templateoptions}
  \end{element}

  \begin{element}{itemize item}\yes\yes\yes
    \colorfontparents{item}
    This template (with |item| instead of |items|) governs how the
    marker in front of a first-level item is typeset. ``First-level''
    refers to the level of nesting. See the |itemize items| template
    for the \meta{options} that may be given.

    When the template is inserted, the \beamer-font and -color
    |itemize item| is installed. Typically, the font is ignored by the
    template as some special symbol is drawn anyway, by the font may
    be important if an optional argument is given to the
    |\item| command as in |\item[First]|.

    The font and color inherit from the |item| font and color, which
    are explained at the end of this section.
  \end{element}

  \begin{element}{itemize subitem}\yes\yes\yes
    \colorfontparents{subitem}
    Like |itemize item|, only for second-level items. An
    item of an itemize inside an enumerate counts a second-level item.
  \end{element}

  \begin{element}{itemize subsubitem}\yes\yes\yes
    \colorfontparents{subsubitem}
    Like |itemize item|, only for third-level items.
  \end{element}
\end{environment}




\begin{environment}{{enumerate}\opt{|[<|\meta{default overlay specification}|>]|}\oarg{mini template}} 
  Used to display a list of items that are ordered.  Inside the
  environment, use an |\item| command for each topic. By default,
  before each item increasing Arabic numbers  followed by a dot are
  printed (as in ``1.'' and ``2.''). This can be changed by specifying
  a different template,  see
  Section~\ref{section-template-enumerate}.

  The first optional argument \meta{default overlay specification} has
  exactly the same effect as for the |itemize| environment. It is
  ``detected'' by the opening |<|-sign in the \meta{default overlay
    specification}. Thus, if there is only one optional argument and
  if this argument does not start with |<|, then it is considered to
  be a \meta{mini template}. 

  The syntax of the \meta{mini template} is the same as
  the syntax of mini templates in the |enumerate| package (you do not
  need to include the 
  |enumerate| package, this is done automatically). Roughly spoken,
  the text of the \meta{mini template} is printed before each item,
  but any occurrence of a |1| in the mini template is replaced by the
  current item number, an occurrence of the letter |A| is replaced by
  the $i$th letter of the alphabet (in uppercase) for the $i$th item,
  and the letters |a|, |i|, and |I| are replaced by the corresponding
  lowercase letters, lowercase Roman letters, and uppercase Roman
  letters, respectively. So the mini template |(i)| would yield the
  items (i), (ii), (iii), (iv), and so on. The mini template |A.)|
  would yield the items A.), B.), C.), D.) and so on. For more details
  on the possible mini templates, see the documentation of the
  |enumerate| package. Note that there is also a template that governs
  the appearance of the mini template (for example, to change its
  color), see Section~\ref{section-template-enumerate}.
  
  \example
\begin{verbatim}
\begin{enumerate}
\item This is important.
\item This is also important.
\end{enumerate}

\begin{enumerate}[(i)]
\item First Roman point.
\item Second Roman point.
\end{enumerate}

\begin{enumerate}[<+->][(i)]
\item First Roman point.
\item Second Roman point, uncovered on second slide.
\end{enumerate}
\end{verbatim}

  \articlenote
  To use the \meta{mini template}, you have to include the package
  |enumerate|.  

  \lyxnote
  The same constraints as for |itemize| apply.
  
  \begin{element}{enumerate items}\semiyes\no\no
    Similar to |itemize items|, this template is a parent template,
    whose children are |enumerate item|, |enumerate subitem|,
    |enumerate subsubitem|, and |enumerate mini template|. These
    templates govern how the text (the number) of an enumeration is
    typeset. 
    
    \begin{templateoptions}
      \itemoption{default}{}
      The default enumeration marker uses the scheme 1., 2., 3.\ for
      the first level, 1.1, 1.2, 1.3 for the second level and 1.1.1,
      1.1.2, 1.1.3 for the third level. 
      \itemoption{circle}{}
      Places the numbers inside little circles. The colors are taken
      from |item projected| or |subitem projected| or
      |subsubitem projected|.
      \itemoption{square}{}
      Places the numbers on little squares.
      \itemoption{ball}{}
      ``Projects'' the numbers onto little balls.
    \end{templateoptions}
  \end{element}

  \begin{element}{enumerate item}\yes\yes\yes
    This template governs how the number in front of a first-level
    item is typeset. The level here refers to the level of enumeration
    nesting only. Thus an enumerate inside an itemize is a first-level
    enumerate (but it uses the second-level
    |itemize/enumerate body|). 

    When the template is inserted, the \beamer-font and -color
    |enumerate item| are installed.

    The following command is useful for this template:
    \begin{templateinserts}
      \iteminsert{\insertenumlabel}
      inserts the current number of the top-level enumeration (as an
      Arabic number). This insert is also available in the next two
      templates. 
    \end{templateinserts}
  \end{element}

  \begin{element}{enumerate subitem}\yes\yes\yes
    Like |enumerate item|, only for second-level items. 

    \begin{templateinserts}
      \iteminsert{\insertsubenumlabel}
      inserts the current number of the second-level enumeration (as an
      Arabic number). 
    \end{templateinserts}

    \example |\setbeamertemplate{enumerate subitem}{\insertenumlabel-\insertsubenumlabel}|
  \end{element}

  \begin{element}{enumerate subsubitem}\yes\yes\yes
    Like |enumerate item|, only for third-level items. 

    \begin{templateinserts}
      \iteminsert{\insertsubsubenumlabel}
      inserts the current number of the second-level enumeration (as an
      Arabic number). 
    \end{templateinserts}
  \end{element}

  \begin{element}{enumerate mini template}\yes\yes\yes
    This template is used to typeset the number that arises from a
    mini template.

    \begin{templateinserts}
      \iteminsert{\insertenumlabel}
      inserts the current number rendered by this mini template. For
      example, if the \meta{mini template} is |(i)| and this command
      is used in the fourth item, |\insertenumlabel| would yield
      |(iv)|.
    \end{templateinserts}
  \end{element}
\end{environment}

The following templates govern how the \emph{body} of an |itemize| or
an |enumerate| is typeset.
\begin{element}{itemize/enumerate body begin}\yes\no\no
  This template is inserted at the beginning of a first-level
  |itemize| or |enumerate| environment. Furthermore, before this
  template is inserted, the \beamer-font and -color
  |itemize/enumerate body| is used.
\end{element}
\begin{element}{itemize/enumerate body end}\yes\no\no
  This template is inserted at the end of a first-level
  |itemize| or |enumerate| environment.
\end{element}
There exist corresponding templates like
|itemize/enumerate subbody being| from second- and third-level itemize
or enumerates.

\begin{element}{items}\semiyes\no\no
  This template is a parent template of |itemize items| and
  |enumerate items|.
  \example |\setbeamertemplate{items}[circle]| will cause all items in
  |itemize| or |enumerate| environments to become circles (of the
  appropriate size, color, and font).
\end{element}


\label{section-descriptions}
  
\begin{environment}{{description}\opt{|[<|\meta{default overlay specification}|>]|}\oarg{long text}} 
  Like |itemize|, but used to display a list that explains or defines
  labels. The width of \meta{long text} is used to set the
  indentation. Normally, you choose the widest label in the
  description and copy it here. If you do not give this argument, the
  default width is used, which can be changed using |\setbeamermargin|
  with the argument |descriptionwidth=|\meta{width}.

  As for |enumerate|, the \meta{default overlay specification} is
  detected by an opening~|<|. The effect is the same as for
  |enumerate| and |itemize|.
  \example
\begin{verbatim}
\begin{description}
\item[Lion] King of the savanna.
\item[Tiger] King of the jungle.
\end{description}

\begin{description}[longest label]
\item<1->[short] Some text.
\item<2->[longest label] Some text.
\item<3->[long label] Some text.
\end{description}
\end{verbatim}

  \example The following has the same effect as the previous example:
\begin{verbatim}
\begin{description}[<+->][longest label]
\item[short] Some text.
\item[longest label] Some text.
\item[long label] Some text.
\end{description}
\end{verbatim}

  \lyxnote
  Since you cannot specify the optional argument in \LyX, if you wish
  to specify the width, you may wish to use the following command
  shortly before the environment:

  |\setbeamermargin{descriptionwidthof}|\marg{text}
  
  \begin{element}{description item}\yes\yes\yes
    This template is used to typeset the description items. When this
    template is called, the \beamer-font and -color |description item|
    are installed.
    
    \begin{templateoptions}
      \itemoption{default}{}
      By default, the description item text is just inserted without
      any modification.
    \end{templateoptions}

    The main insert that is useful inside this template is:
    \begin{templateinserts}
      \iteminsert{\insertdescriptionitem} inserts the text of the
      current description item.
    \end{templateinserts}
  \end{element}
\end{environment}


\begin{command}{\beamersetusedescriptionitemofwidthas\marg{long text}}
  This command overrides the default width of the
  description label by the width of \meta{long text} for the current
  \TeX\ group. You should only use this command if, for some reason or
  another, you cannot give the \meta{long text} as an argument to the
  |description| environment. This happens, for example, if you create a
  |description| environment in \LyX.

  \example
\begin{verbatim}
\usedescriptionitemofwidthas{longest label}
\begin{description}
\item<1->[short] Some text.
\item<2->[longest label] Some text.
\item<3->[long label] Some text.
\end{description}
\end{verbatim}
\end{command}



In order to simplify changing the color or font of items, the
different kinds of items inherit form or just use the following
``general'' \beamer-color and fonts:

\begin{element}{item}\no\yes\yes
\end{element}

\begin{element}{item projected}\no\yes\yes
\end{element}

\begin{element}{subitem}\no\yes\yes
\end{element}

\begin{element}{subitem projected}\no\yes\yes
\end{element}

\begin{element}{subsubitem}\no\yes\yes
\end{element}

\begin{element}{subsubitem projected}\no\yes\yes
\end{element}




\subsubsection{Hilighting}

The \beamer\ class predefines commands and environments for
hilighting text. Using these commands makes is easy to change the
appearance of a document by changing the theme. 


\begin{command}{\structure\sarg{overlay specification}\marg{text}}
  The given text is marked as part of the structure, that is, it is
  supposed to help the audience see the structure of your
  presentation. If the \meta{overlay specification} is present, the
  command only has an effect on the specified slides.
  \example|\structure{Paragraph Heading.}|

  Internally, this command just puts the \emph{text} inside a
  |structureenv| environment.
 
  \articlenote
  Structure text is typeset as bold text. This can be changed by
  modifying the templates.

  \lyxnote
  You need to use \TeX-mode to insert this command.

  \begin{element}{structure}\no\yes\yes
    This color/font is used when structured text is typeset, but it is
    also widely used as a base for many other colors including the
    headings of blocks, item buttons, and titles. In most color
    themes, the colors for navigational elements in the headline or
    the footline are derived from the foreground color of
    |structure|. By changing the structure color you can easily change
    the ``basic color'' of your presentation, other than the color of
    normal text. See also the related color |local structure| and the
    related font |tiny structure|.
    
    Inside the |\structure| command, the background of the color is
    ignored, but this is not necessarily true for elements that
    inherit their color from |structure|. There is no template
    |structure|, use |structure begin| and |structure end| instead.
  \end{element}
 
  \begin{element}{local structure}\no\yes\no
    This color should be used to typeset structural elements that change
    their color according to the ``local environment.'' For example, the
    color of an item ``button'' in an |itemize| environment changes its
    color according to circumstances. If it is used inside an example
    block, it should have the |example text| color; if it is currently
    ``alerted'' it should have the |alerted text| color. This color
    will setup by certain environments to have the color that should be
    used to typset things like item buttons. Since the color used for
    items, |item|, inherits from this color by default, items
    automatically change their color according to the current
    situation.

    If you write your own environment in which the item buttons are
    similar structural elements should have a different color, you
    should change the color |local structure| inside these
    environments. 
  \end{element}
  
  \begin{element}{tiny structure}\no\no\yes
    This special font is used for ``tiny'' structural text. Basically,
    this font should be used whenever a structural element uses a tiny
    font. The idea is that the tiny versions of the |structure| font
    often are not suitable. For example, it is often necessary to use a
    boldface version for them. Also, one might wish to have serif smallcaps
    structural text, but still retain normal sans-serif tiny structural
    text.
  \end{element}
\end{command}

\begin{environment}{{structureenv}\sarg{overlay specification}}
  Environment version of the |\structure| command.

  \begin{element}{structure begin}\yes\no\no
    This text is inserted at the beginning of a |structureenv|
    environment.

    \begin{templateoptions}
      \itemoption{default}{}

      \articlenote
      The text is typeset in boldface.
    \end{templateoptions}
  \end{element}

  \begin{element}{structure end}\yes\no\no
    This text is inserted at the end of a |structureenv| environment.
  \end{element}
\end{environment}


\begin{command}{\alert\sarg{overlay specification}\marg{hilighted text}}
  The given text is hilighted, typically be coloring the text red. If
  the \meta{overlay specification} is present, the command only has an
  effect on the specified slides.
  \example |This is \alert{important}.|

  Internally, this command just puts the \emph{hilighted text} inside
  an |alertenv|.
  
  \articlenote
  Alerted text is typeset as emphasized text. This can be changed by
  modifying the templates, see below.

  \lyxnote
  You need to use \TeX-mode to insert this command (which is not very
  convenient).

  \begin{element}{alerted text}\no\yes\yes
    This color/font is used when alerted text is typeset. The
    background is currently ignored. There is no template
    |alerted text|, rather there are templates |alerted text begin|
    and |alerted text end| that are inserted before and after alerted
    text.
  \end{element}
\end{command}

\begin{environment}{{alertenv}\sarg{overlay specification}}
  Environment version of the |\alert| command.

  \begin{element}{alerted text begin}\yes\no\no
    This text is inserted at the beginning of a an |alertenv|
    environment.

    \begin{templateoptions}
      \itemoption{default}{}

      \beamernote
      This changes the color |local structure| to |alerted text|. This
      causes things like buttons or items to be colored in the same
      color as the alerted text, which is often visually pleasing. See
      also the |\structure| command.

      \articlenote
      The text is emphasized.
    \end{templateoptions}
  \end{element}

  \begin{element}{alerted text end}\yes\no\no
    This text is inserted at the end of an |alertenv| environment.
  \end{element}
\end{environment}




\subsubsection{Block Environments}
\label{predefined}

The \beamer\ class predefines an environment for typesetting a
``block'' of text that has a heading. The appearence of the block is
governed by a template.

\begin{environment}{{block}\sarg{action specification}\marg{block
      title}\sarg{action specification}}
  Only one \meta{action specification} may be given.
  Inserts a block, like a definition or a theorem, with the title
  \meta{block title}. If the \meta{action specification} is present,
  the given actions are taken on the specified slides, see
  Section~\ref{section-action-specifications}. In the example, the 
  definition is shown only from slide 3 onward.
  \example
\begin{verbatim}
  \begin{block}<3->{Definition}
    A \alert{set} consists of elements.
  \end{block}
\end{verbatim}

  \articlenote
  The block name is typeset in bold.

  \lyxnote
  The argument of the block must (currently) be given in
  \TeX-mode. More precisely, there must be an opening brace in
  \TeX-mode and a closing brace in \TeX-mode around it. The text
  in between can also be typeset using \LyX. I hope to get rid of this
  some day.
\end{environment}


\begin{environment}{{alertblock}\sarg{action specification}\marg{block
title}\sarg{action specification}} 
  Inserts a block whose title is hilighted. Behaves like the |block|
  environment otherwise.
  \example
\begin{verbatim}
  \begin{alertblock}{Wrong Theorem}
    $1=2$.
  \end{alertblock}
\end{verbatim}

  \articlenote
  The block name is typeset in bold and is emphasized.

  \lyxnote
  Same applies as for |block|.
\end{environment}

\begin{environment}{{exampleblock}\sarg{action
specification}\marg{block title}\sarg{overlay specification}} 
  Inserts a block that is supposed to be an example. Behaves like the
  |block| environment otherwise.
  
  \example In the following example, the block is completely
  suppressed on the first slide (it does not even occupy any space).
\begin{verbatim}
  \begin{exampleblock}{Example}<only@2->
    The set $\{1,2,3,5\}$ has four elements.
  \end{exampleblock}
\end{verbatim}

  \articlenote
  The block name is typeset in italics.

  \lyxnote
  Same applies as for |block|.
\end{environment}

\lyxnote
Overlay specifications must be given right at the beginning of the
environments and in \TeX-mode.



\subsubsection{Theorem Environments}
\label{section-theorems}

The \beamer\ class predefines several environments, like |theorem| or
|definition| or |proof|, that you can use to typeset things like,
well, theorems, definitions, or proofs. The complete list is the
following:  |theorem|, |corollary|, |definition|,
|definitions|, |fact|, |example|, and |examples|. The following German
block environments are also predefined: |Problem|, |Loesung|,
|Definition|, |Satz|, |Beweis|, |Folgerung|, |Lemma|, |Fakt|,
|Beispiel|, and |Beispiele|.

Here is a typical example on how to use them:

\begin{verbatim}
\begin{frame}
  \frametitle{A Theorem on Infinite Sets}

  \begin{theorem}<1->
    There exists an infinite set.
  \end{theorem}

  \begin{proof}<2->
    This follows from the axiom of infinity.
  \end{proof}

  \begin{example}<3->[Natural Numbers]
    The set of natural numbers is infinite.
  \end{example}
\end{frame}
\end{verbatim}

In the following, only the English versions are discussed. The German
ones behave  analogously.

\begin{environment}{{theorem}\sarg{action
  specification}\oarg{additional text}\sarg{action specification}}
  Inserts a theorem. Only one \meta{action specification} may be
  given. If present, the \meta{additional text} is shown behind the
  word ``Theorem'' in rounded brackets (although this can be changed by
  the template).

  The appearance of the theorem is governed by
  templates, see Section~\ref{section-theorems-templates} for details
  on how to change these. Every theorem is put into a |block|
  environment, thus the templates for blocks also apply.

  The theorem style (a concept from |amsthm|) used for this
  environment is |plain|. In this style, the body of a theorem should
  be typeset in italics. The head of the theorem should be typeset in
  a bold font, but this is usually overruled by the templates.

  If the option |envcountsect| is given either as class option in one
  of the |presentation| modes or as an option to the package
  |beamerarticle| in |article| mode, then the numbering of the
  theorems is local to each section with the section number prefixing
  the theorem number; otherwise they are numbered consecutively
  throughout the presentation or article. I recommend using this
  option in |article| mode.

  By default, no theorem numbers are shown in the |presentation| modes.

  \example
\begin{verbatim}
\begin{theorem}[Kummer, 1992]
  If $\#^_A^n$ is $n$-enumerable, then $A$ is recursive.
\end{theorem}

\begin{theorem}<2->[Tantau, 2002]
  If $\#_A^2$ is $2$-fa-enumerable, then $A$ is regular.
\end{theorem}
\end{verbatim}

  \lyxnote
  If present, the optional argument and the action specification must
  be given in \TeX-mode at the beginning of the environment.
\end{environment}

The environments \declare{|corollary|}, \declare{|fact|}, and
\declare{|lemma|} behave exactly the same way.

\begin{classoption}{{envcountsect}}
  Causes theorems, definitions, and so on to be numbered locally to
  each section. Thus, the first theorem of the second section would be
  Theorem~2.1 (assuming that there are no definitions, lemmas, or
  corollaries earlier in the section).
\end{classoption}

\begin{environment}{{defintion}\sarg{action
      specification}\oarg{additional text}\sarg{action specification}}
  Behaves like the |theorem| environment, except that the theorem
  style |definition| is used. In this style, the body of a theorem is
  typeset in an upright font.
\end{environment}

The environment \declare{|definitions|} behaves exactly the same way.

\begin{environment}{{example}\sarg{action
      specification}\oarg{additional text}\sarg{action specification}}
  Behaves like the |theorem| environment, except that the theorem
  style |example| is used. A side-effect of using this theorem style
  is that the \meta{environment contents} is put in an |exampleblock|
  instead of a |block|.
\end{environment}

The environment \declare{|examples|} behaves exactly the same way.

\beamernote
The default template for typesetting theorems suppresses the theorem
number, even if this number is ``available'' for typesetting (which it
is by default in all predefined environments; but if you define your
own environment using |\newtheorem*| no number will be available).

\articlenote
In |article| mode, theorems are automatically numbered. By specifying
the class option |envcountsect|, theorems will be numbered locally to each
section, which is usually a good idea, except for very short
articles.

\begin{environment}{{proof}\sarg{action specification}\oarg{proof
name}\sarg{action specification}}
  Typesets a proof. If the optional \meta{proof name} is given, it
  completely replaces the word ``Proof.'' This is different from
  normal theorems, where the optional argument is shown in brackets.

  At the end of the theorem, a |\qed| symbol is shown, except if you
  say |\qedhere| earlier in the proof (this is exactly as in
  |amsthm|). The default |\qed| symbol is an open rectangle. To
  completely suppress the symbol, write |\def\qedsymbol{}| in 
  your preamble. To get an closed rectangle, say
\begin{verbatim}
\useqedsymboltemplate{\color{beamerstructure}\vrule width1.5ex height1.5ex depth0pt}
\end{verbatim}

  If you use |babel| and a different language, the text ``Proof'' is
  replaced by whatever is appropriate in the selected language.

  \example
\begin{verbatim}
\begin{proof}<2->[Sketch of proof]
  Suppose ...
\end{proof}
\end{verbatim}
\end{environment}

You can define new environments using the following command:

\begin{command}{\newtheorem\opt{|*|}\marg{environment name}\oarg{numbered same
      as}\marg{head text}\oarg{number within}}
  This command is used exactly the same way as in the |amsthm| package
  (as a matter of fact, it is the command from that package), see its
  documentation. The only difference is that environments declared using
  this command are overlay-specification-aware in \beamer\ and that,
  when typeset, are typeset according to \beamer's templates.

  \articlenote
  Environments declared using this command are also
  overlay-specification-aware in |article| mode.

  \example |\newtheorem{observation}[theorem]{Observation}|
\end{command}

You can also use |amsthm|'s command |\newtheoremstyle| to define new
theorem styles. Note that the default template for theorems will
ignore any head font setting, but will honor the body font setting.

If you wish to define the environments like |theorem| differently (for
example, have it numbered within each subsection), you can use the
following class option to disable the definition of the predefined
environments: 

\begin{classoption}{{notheorems}}
  Switches off the definition of default blocks like |theorem|, but
  still loads |amsthm| and makes theorems  
  overlay-specificiation-aware.
\end{classoption}

The option is also available as a package option for
|beamerarticle| and has the same effect.

\articlenote
In the |article| version, the package |amsthm| sometimes clashes with
the document class. In this case you can use the following option,
which is once more available as a class option for \beamer\ and as a
package option for |beamerarticle|, to switch off the loading of
|amsthm| altogether. 

\begin{classoption}{{noamsthm}}
  Does not load |amsthm| and also not |amsmath|. Environments like
  |theorem| or |proof| will not be available.
\end{classoption}




\subsubsection{Framed Text}

In order to draw a frame (a rectangle) around some text, you can use
\LaTeX s standard command |\fbox| and also |\frame| (inside a \beamer\
frame, the |\frame| command changes its meaning to the normal \LaTeX\
|\frame| command). More frame types are offered by the
package |fancybox|, which defines the following commands:
|\shadowbox|, |\doublebox|, |\ovalbox|, and |\Ovalbox|. Please consult
the \LaTeX\ Companion for details on how to use these commands.

The \beamer\ class also defines an environment for creating boxes:

\begin{environment}{{beamerboxesrounded}\oarg{options}\marg{head}}
  The text inside the environment is framed by a rectangular area with
  rounded corners. The background of the rectangular area is filled
  with a certain color, which depends on the current color scheme (see
  below). If the \meta{head} is not empty, \meta{head} is drawn in the
  upper part of the box in a different color, which also depends on
  the scheme. The following options can be given:
  \begin{itemize}
  \item \declare{|scheme=|\meta{name}} causes the color scheme \meta{name} to be
    used. A color scheme must previously be defined using the command
    |\beamerboxesdeclarecolorscheme|.
  \item \declare{|width=|\meta{dimension}} causes the width of the text inside
    the box to be the specified \meta{dimension}. By default, the
    |\textwidth| is used. Note that the box will protrude 4pt to the
    left and right.
  \item \declare{|shadow=|\meta{true or false}}. If set to |true|, a shadow will
    be drawn.    
  \end{itemize}
  A color scheme dictates the background colors used in the head part
  and in the body of the box. If no \meta{head} is given, the head
  part is completely suppressed.
  \example
\begin{verbatim}
\begin{beamerboxesrounded}[scheme=alert,shadow=true]{Theorem}
  $A = B$.
\end{beamerboxesrounded}
\end{verbatim}

  \articlenote
  This environment is not available in |article| mode.
\end{environment}

\begin{command}{\beamerboxesdeclarecolorscheme\marg{scheme
      name}\marg{head color}\marg{body color}}
  Declares a color scheme for later use in a |beamerboxesrounded|
  environment.
  \example |\beamerboxesdeclarecolorscheme{alert}{red}{red!15!averagebackgroundcolor}|

  \articlenote
  This command is not available in |article| mode.
\end{command}



\subsubsection{Figures and Tables}

You can use the standard \LaTeX\ environments |figure| and
|table| much the same way you would normally use them. However,
any placement specification will be ignored. Figures and tables are
immediately inserted where the environments start. If there are too
many of them to fit on the frame, you must manually split them among
additional frames or use the |allowframebreaks| option.

\example
\begin{verbatim}
\begin{frame}
  \begin{figure}
    \pgfuseimage{myfigure}
    \caption{This caption is placed below the figure.}
  \end{figure}

  \begin{figure}
    \caption{This caption is placed above the figure.}
    \pgfuseimage{myotherfigure}
  \end{figure}
\end{frame}
\end{verbatim}

You can adjust how the figure and table captions are typeset by
changing the corresponding template, see
Section~\ref{section-template-caption}. 





\subsubsection{Splitting a Frame into Multiple Columns}

The \beamer\ class offers several commands and environments for
splitting (perhaps only part of) a frame into multiple columns. These
commands have nothing to do with \LaTeX's commands for creating
columns. Columns are especially useful for placing a graphic next to a
description/explanation.

The main environment for creating columns is called |columns|. Inside
this environment, you can either place several |column| environments,
each of which creates a new column, or use the |\column| command to
create new columns.

\begin{environment}{{columns}\oarg{options}}
  A multi-column area. Inside the environment you should place only
  |column| environments or |\column| commands (see below). The
  following \meta{options} may be given: 
  \begin{itemize}
  \item
    \declare{|b|} will cause the bottom lines of the columns to be
    vertically aligned.
  \item
    \declare{|c|} will cause the columns to be centered vertically
    relative to each other. Default, unless the global option
    |t| is used. 
  \item
    \declare{|onlytextwidth|} is the same as |totalwidth=\textwidth|.
  \item
    \declare{|t|} will cause the first lines of the columns to be
    aligned. Default if global option |t| is used.
  \item
    \declare{|totalwidth=|\meta{width}} will cause the columns to occupy
    not the whole page width, but only \meta{width}, all told.
  \end{itemize}
    
  \example
\begin{verbatim}
\begin{columns}[t]
  \begin{column}{5cm}
    Two\\lines.
  \end{column}
  \begin{column}{5cm}
    One line (but aligned).
  \end{column}
\end{columns}
\end{verbatim}
  
  \example
\begin{verbatim}
\begin{columns}[t]
  \column{5cm}
    Two\\lines.

  \column{5cm}
    One line (but aligned).
\end{columns}
\end{verbatim}

  \articlenote
  This environment is ignored in |article| mode.
  
  \lyxnote
  Use ``Columns'' or ``ColumnsTopAligned'' to create a |columns|
  environment. To pass options, insert them in \TeX-mode right at the
  beginning of the environment in square brackets.
\end{environment}

To create a column, you can either use the |column| environment or the
|\column| command. 

\begin{environment}{{column}\oarg{placement}\marg{column width}}
  Creates a single column of width \meta{column width}. The vertical
  placement of the enclosing |columns| environment can be overruled by
  specifying a specific \meta{placement} (|t| for top, |c| for
  centered, and |b| for bottom). 

  \example The following code has the same effect as the above examples:
\begin{verbatim}
\begin{columns}
  \begin{column}[t]{5cm}
    Two\\lines.
  \end{column}
  \begin{column}[t]{5cm}
    One line (but aligned).
  \end{column}
\end{columns}
\end{verbatim}
  \articlenote
  This command is ignored in |article| mode.

  \lyxnote
  The ``Column'' styles insert the command version, see below.
\end{environment}

\begin{command}{{\column}\oarg{placement}\marg{column width}}
  Starts a single column. The parameters and options are the same as
  for the |column| environment. The column automatically ends with the
  next occurrence of |\column| or of a |column| environment or of the
  end of the current |columns| environment.

  \example 
\begin{verbatim}
\begin{columns}
  \column[t]{5cm}
    Two\\lines.
  \column[t]{5cm}
    One line (but aligned).
\end{columns}
\end{verbatim}
  \articlenote
  This command is ignored in |article| mode.

  \lyxnote
  In a ``Column'' style, the width of the column must be given as
  normal text, not in \TeX-mode.
\end{command}



\subsubsection{Positioning Text and Graphics Absolutely}

Normally, \beamer\ uses \TeX's normal typesetting mechanism to
position text and graphics on the page. In certain situation you may
instead wish a certain text or graphic to appear at a
page position that is specified \emph{absolutely}. This means that the
position is specified relative to the upper left corner of the slide.

The package |textpos| provides several commands for positioning text
absolutely and it works together with \beamer. When using this
package, you will typically have to specify the options |overlay| and
perhaps |absolute|. For details on how to use the package, please see
its documentation.




\subsubsection{Verse, Quotations, Quotes}

\LaTeX\ defines three environments for typesetting quotations and
verses: |verse|, |quotation|, and |quote|. These environments are also
available in the \beamer\ class, where they are
overlay-specification-aware. If an overlay specification is given, the
verse or quotation is shown only on the specified slides and is
covered otherwise. The difference between a |quotation| and a |quote|
is that the first has paragraph indentation, whereas the second does
not. 

Unlike the standard \LaTeX\ environments, in \beamer\ these
environments do not only change the left and right margins, but also
the font: A verse is typeset using an italic serif font, quotations
and quotes are typeset using an italic font (whether serif or
sans-serif depends on the standard document font). To change this, you
can adjust the templates for these environments. 


\subsubsection{Footnotes}

First a word of warning: Using footnotes is usually not a good
idea. They disrupt the flow of reading.

You can use the usual |\footnote| command. It has been augmented to
take an additional option, for placing footnotes at the frame
bottom instead of at the bottom of the current minipage.

\begin{command}{\footnote\sarg{overlay
      specification}\oarg{options}\marg{text}} 
  Inserts a footnote into the current frame. Footnotes will always be
  shown at the bottom of the current frame; they will never be
  ``moved'' to other frames. As usual, one can give a number as
  \meta{options}, which will cause the footnote to use that
  number. The \beamer\ class adds one additional option:
  \begin{itemize}
  \item \declare{|frame|} causes the footnote to be shown at the
    bottom of the frame. This is normally the default behavior anyway,
    but in minipages and certain blocks it makes a difference. In a
    minipage, the footnote is usually shown as part of the minipage
    rather than as part of the frame.
  \end{itemize}

  If an \meta{overlay specification} is given, this causes the
  footnote \meta{text} to be shown only on the specified slides. The
  footnote symbol in the text is shown on all slides.

  \example |\footnote{On a fast machine.}|
  \example |\footnote[frame,2]{Not proved.}|
  \example |\footnote<.->{Der Spiegel, 4/04, S.\ 90.}|
\end{command}

You can change the way footnotes are typeset by changing the footnote
templates, see Section~\ref{section-templates-footnotes}


%%% Local Variables: 
%%% mode: latex
%%% TeX-master: "beameruserguide"
%%% End: 
