% $Header$

\documentclass{beamer}

% Diese Datei enth�lt eine L�sungsvorlage f�r:

% - Das Vorstellen des n�chsten Sprechers.
% - Vortragsl�nge von ca. zwei Minuten.
% - Aussehen des Vortrags ist verschn�rkelt/dekorativ.



% Copyright 2004 by Till Tantau <tantau@users.sourceforge.net>.
%
% In principle, this file can be redistributed and/or modified under
% the terms of the GNU Public License, version 2.
%
% However, this file is supposed to be a template to be modified
% for your own needs. For this reason, if you use this file as a
% template and not specifically distribute it as part of a another
% package/program, I grant the extra permission to freely copy and
% modify this file as you see fit and even to delete this copyright
% notice. 


\setbeamertemplate{background canvas}[vertical shading][bottom=white,top=structure.fg!25]
% Oder was auch immer.

\usetheme{Warsaw}
\setbeamertemplate{headline}{}
\setbeamertemplate{footline}{}
\setbeamersize{text margin left=0.5cm}
  
\usepackage[german]{babel}
% Oder was auch immer

\usepackage[latin1]{inputenc}
% Oder was auch immer

\usepackage{times}
\usepackage[T1]{fontenc}
% Oder was auch immer. Zu beachten ist, das Font und Encoding passen
% m�ssen. Falls T1 nicht funktioniert, kann man versuchen, die Zeile
% mit fontenc zu l�schen.



\begin{document}

\begin{frame}{Name der/des Vortragenden}{�ber unsere(n) n�chste(n) Vortragende(n)}

  \begin{itemize}
  \item
    Derzeitige Stellung(en) von Name der/des Vortragenden
    
    % Beispiele:
    \begin{itemize}
    \item
      Professor f�r Mathematik, Universit�t Irgendwo.
    \item
      Juniorpartner bei Firma X.
    \item
      Sprecher der Organisation X / des Projekts X.
    \end{itemize}
  \item
    Erfahrungen und Erfolge
    % Optional. Sollte nur gebracht werden, wenn es angemessen
    % erscheint, dem Vortragenden leicht zu schmeicheln; was
    % beispielsweise bei eingeladenen Vortragenden der Fall ist. 
    % Die Unterpunkte sollten den Sprecher interessant und kompetent
    % erscheinen lassen.

    % Beispiele:
    \begin{itemize}
    \item
      Falls passend, akademischer Grad
    \item
      Derzeitige und/oder vorherige Stellungen, eventuell mit Daten
    \item
      Publikationen (eventuell lediglich die Anzahl)
    \item
      Auszeichnungen, Preise
    \end{itemize}
  \item
    In Bezug auf den heutigen Vortrag
    % Optional. Sollte benutzt werden, um spezielle Erfahrungen /
    % spezielles Wissen des Vortragenen speziell in Bezug auf den
    % Vortrag aufzuzeigen -- falls diese Erfahrungen sich nicht
    % bereits aus obigen Punkten ableiten lassen.

    % Beispiele:
    \begin{itemize}
    \item
      Expertin/Experte, die/der seit X Monaten/Jahren in dem
      Gebiet/Projekt arbeitet.
    \item
      Wird ihre/seine Forschung / die Forschung der Gruppe/Firma vorstellen.
    \item
      Wird einen Projektbericht/Projektstatus vorstellen.
    \end{itemize}
  \end{itemize}  
\end{frame}

\end{document}


