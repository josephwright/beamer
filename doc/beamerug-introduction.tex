
% Copyright 2003, 2004 by Till Tantau <tantau@users.sourceforge.net>.
%
% This program can be redistributed and/or modified under the terms
% of the GNU Public License, version 2.

\section{Introduction}

This user's guide explains the functionality of the \beamer\ class.
It is a \LaTeX\ class that allows you to create a presentation with a
projector. It can also be used to create slides. It behaves 
similarly to other packages like \textsc{prosper}, but has the
advantage that it works together directly with |pdflatex|, but
also with |dvips|.

\subsection{Getting Started with the Beamer Class and \LaTeX/pdf\LaTeX}

To use the \beamer\ class together with |latex| or |pdflatex|, proceed
as follows: 
\begin{enumerate}
\item
  Specify |beamer| as document class instead of
  |article|.
\item
  Structure your \LaTeX\ text using |section| and
  |subsection| commands.
\item
  Place the text of the individual slides inside |frame|
  commands.
\item
  Run |pdflatex| on the text (or |latex|,
  |dvips|, and |ps2pdf|).
\end{enumerate}

To get jump-started try opening one of the |.tex| files in the
|solutions| directory.

The \beamer\ class has several useful features: You don't need any
external programs to use it other than |pdflatex|, but it works
also with |dvips|. You can easily and intuitively create
sophisticated overlays. Finally, you can easily change the whole slide
theme or only parts of it. The following code shows a typical usage of
the class.

\begin{verbatim}
\documentclass{beamer}

\usepackage{beamerthemesplit}

\title{Example Presentation Created with the Beamer Package}
\author{Till Tantau}
\date{\today}

\begin{document}

\begin{frame}
  \titlepage
\end{frame}

\section*{Outline}
\frame{\tableofcontents}

\section{Introduction}
\subsection{Overview of the Beamer Class}

\begin{frame}
  \frametitle{Features of the Beamer Class}

  \begin{itemize}
  \item<1-> Normal LaTeX class.
  \item<2-> Easy overlays.
  \item<3-> No external programs needed.      
  \end{itemize}
\end{frame}
\end{document}
\end{verbatim}

Run |pdflatex| on this code (twice) and then use, for example, the
Acrobat Reader to present the resulting |.pdf| file in a
presentation. You can also, alternatively, use |dvips|; see
Section~\ref{section-postscript} for details.

As can be seen, the text looks almost like a normal \LaTeX\ text. The
main difference is the usage of the |frame| environment. This
environment will typically show its contents on a single
slide. However, in case you use overlay commands inside a frame, a 
single frame command may produce several slides. An example is the
last frame in the above example. There, the |\item| commands
are followed by \emph{overlay specifications} like |<1->|,
which means ``from slide 1 on.'' Such a specification causes the item
to be shown only on the specified slides of the frame (see
Section~\ref{section-overlay} for details). In the above example, a
total of five slides are produced: a title page slide, an outline
slide, a slide showing only the first of the three items, a slide
showing the first two of them, and a slide showing all three items.
 
To structure your text, you can use the commands |\section| and
|\subsection|. These commands will not only create entries in the
table of contents, but will also in the navigation bars.


\subsection{Getting Started with the Beamer Class and \texorpdfstring{\LyX}{LyX}}

Once installed (see Section~\ref{section-installation}), using the
\beamer\ class  together with \LyX\ is quite easy: You open a new file
and choose |beamer| as the document class. It is often even easier to
choose ``New from template'' and to pick a template from one of the
subdirectories of the directory |beamer/solutions|. 

To reproduce the example from the previous subsection in \LyX, proceed
as follows:

\begin{itemize}
\item
  The command |\usepackage{beamerthemesplit}| must be added to the
  preamble. You can edit the preamble using Layout $\rangle$ Document
  $\rangle$ Preamble.
\item
  Typeset the author and date the usual way, using the styles Author
  and Date. The title page will then be created automatically.
\item
  To insert the sections and subsections, use the usual Section and
  Subsection styles.
\item
  To insert the frame containing the table of contents, insert a line
  of style BeginFrame. Since this frame has no title, do not write
  anything on the line with style BeginFrame. Next, insert a line of
  style Standard and use Insert $\rangle$ Insert TOC to insert the
  table of contents. Optionally, end the frame using a line of style
  EndFrame (the following Section style automatically closes the frame).
\item
  To create the last frame, start a new frame using the style
  BeginFrame. Write the frame title on the line having this
  style.
\item
  Use the Itemize style to create the itemized text.
\item
  Add the overlay specifications (the texts like |<1->|) to the items
  by entering \TeX-mode (press on the little \TeX\ icon) and writing
  |<1->|. This \TeX\ text should be placed right at the beginning of
  the item.
\item
  You must end this frame using the style EndFrame (sadly, the end of
  the document and also the beginning of the appendix do not
  automatically end the last frame -- whereas the start of a frame,
  section, part, or subsection does).
\end{itemize}

Now use View $\rangle$ PDF to view the resulting presentation. On a
slow machine, this may take a while. See Section~\ref{section-speedup}
for ways of speeding up the compilation.



\subsection{Solution Templates: Getting Going Quickly}

In the subdirectories of the directory |beamer/solutions| you will
find \emph{solution templates} in different languages. A solution
template is a \TeX-text that ``solves'' a specific problem. Such a
problem might be ``I need to create a 20 minute talk for a
conference'' or ``I want to create a slide that introduces the next
speaker'' or ``I want to create a table that is uncovered piecewise.''
For such a problem, a solution template consists of a mixture of a
template and an example that can be used to solve this particular
problem. Just copy the solution template file (or parts of it) and
freely adjust them to your needs.  

The collecting of \beamer\ solution templates has only begun and
currently there are only very few of them. I hope that in the future
more solutions will become available and I would like to encourage
users of the \beamer\ class to send me solutions they develop. I would
also like to encourage users to help in translating solutions to
languages other than English and German. If you have written a
solution or a translation, please feel free to send it to me.




\subsection{How to Read this User's Guide}

First a word of warning: This text has grown rather wildly and is by
no means perfectly structured or particularly well written. Hopefully
this will change in the future, ideally by this text being superceded
by a good book on creating presentations in \LaTeX\ becoming available.

Till then, this user's guide is both intended as a tutorial and as a
reference guide. If you have not yet installed the package, please read
Section~\ref{section-installation} first. If you do not have much
experience with preparing presentations, 
Section~\ref{section-workflow} might be especially helpful. The later
sections explain the basic usage of the |beamer| class as well as
advanced features. If you wish to adjust the way your presentations
look (for example, if you wish to add a default logo of your
institution to every presentation in the future), please read the
section on customization. 

In this guide you will find the descriptions of all ``public''
commands provided by the |beamer| class. In each such
description, the described command, environment, or option is printed 
in red. Text shown in green is optional and can be left out.

Whenever a \beamer\ solution template exists for a specific problem, this will
be pointed out in the (sub-)section that treats the topic of this
problem. 

You will sometimes find one of the words \textsc{beamer},
\textsc{article}, or \textsc{lyx} in blue in some description of a
command or environment. The first indicates that the description
applies only to ``normal beamer operation in \LaTeX.'' The word
\textsc{article} describes some behaviour that is special for the
|article| mode. The word \textsc{lyx} describes behaviour that is
special when you use \LyX\ to prepare your presentation.  


