\documentclass[serif]{beamer}

% Copyright 2003 by Till Tantau <tantau@cs.tu-berlin.de>.
%
% This program can be redistributed and/or modified under the terms
% of the LaTeX Project Public License Distributed from CTAN
% archives in directory macros/latex/base/lppl.txt.

%
% The purpose of this example is to show how \part can be used to
% organize a lecture.
%

\usepackage{beamerthemeplain}
\usepackage{times}
\usepackage[latin1]{inputenc}

\hypersetup{%
  pdftitle={Beamer Animation Example},%
  pdfauthor={Till Tantau}}

\title{Beamer Animation Example}
\author{Till~Tantau}
\institute{
  Fakult�t f�r Elektrotechnik und Informatik\\
  Technical University of Berlin}


\begin{document}

% View this in acroread with "loop after last page option" in full screen mode.

\newcount\opaqueness
\plainframe{
  \itshape
  \animate<1-30>
  \Large

  \only<1-10>{
  \animatevalue<1-10>{\opaqueness}{100}{10}
  \begin{colormixin}{\the\opaqueness!averagebackgroundcolor}
    \begin{centering}
      \Huge Urfaust\par
    \end{centering}
  \end{colormixin}
  }
 
  \only<11-20>{
  \animatevalue<11-20>{\opaqueness}{100}{10}
  \begin{colormixin}{\the\opaqueness!averagebackgroundcolor}
    \begin{centering}
      Hab nun, ach! die Philosophey,\par
      Medizin und Juristerey \par
      Und leider auch die Theologie\par
      Durchaus studirt mit heisser M�h.\par
  \end{centering}
  \end{colormixin}
  }
 
  \only<21-30>{
  \animatevalue<21-30>{\opaqueness}{100}{10}
  \begin{colormixin}{\the\opaqueness!averagebackgroundcolor}
    \begin{centering}
    Da steh ich nun, ich armer Tohr,\par
    Und binn so klug als wie zuvor.\par
    \end{centering}
  \end{colormixin}}
}

\end{document}


